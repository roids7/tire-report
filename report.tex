% report.tex - 간단한 한글 LaTeX 보고서 템플릿 (XeLaTeX 사용 권장)
\documentclass[10pt,a4paper]{article}

% 한글 설정: fontspec + xeCJK
% 라틴: TeX Live 기본 Latin Modern, 한글: Noto Sans CJK KR (fonts-noto-cjk)
\usepackage{fontspec}
\usepackage{xeCJK}
\defaultfontfeatures{Ligatures=TeX,Scale=MatchLowercase}
\setmainfont{Latin Modern Roman}
\setsansfont{Latin Modern Sans}
\setCJKmainfont{Noto Sans CJK KR}
\setCJKsansfont{Noto Sans CJK KR}

\usepackage{amsmath,amssymb}
\usepackage{graphicx}
\usepackage{caption}
\usepackage{booktabs}
\usepackage{hyperref}
\hypersetup{colorlinks=true, linkcolor=blue, urlcolor=blue}

% 문서 정보
\title{타이어 공정설계 보고서}
\author{홍길동 \\ 학번: 12345678}
\date{\today}

\begin{document}
\maketitle
\begin{abstract}
본 문서는 타이어의 주요 공정(배합, 공압출, 가류)에 대한 조사 결과와 향후 보고서 작성 계획을 정리한 예시 템플릿입니다.
\end{abstract}

\section{서론}

\subsection{배경 및 연구 목표}

승용차 타이어 트레드는 세 가지 고무 컴파운드(크라운 Crown, 베이스 Base, 윙 Wing)를 동시에 성형하는 3중 공압출(tri-composite co-extrusion) 기술로 제조된다. 이 공정은 다음과 같은 특징을 가진다:

\begin{itemize}
  \item \textbf{크라운(Crown)}: 노면과 직접 접촉하여 마모 저항성, 그립감, 미끄럼 방지를 담당
  \item \textbf{베이스(Base)}: 구조적 강도, 발열 특성, 내구성을 담당하는 중간층
  \item \textbf{윙(Wing)}: 코너링 안정성, 소음 감소, 승차감 미세 조정을 담당
\end{itemize}

고무 용융체의 강한 점탄성 거동과 다이 구조의 복잡성으로 인해 유동 분포, 형상, 표면 품질이 복잡하게 결정된다. 특히 세 층 고무의 서로 다른 점탄성 특성(점도, 이완시간, 신장 민감도)으로 인해 전통적인 시험-수정(trial-and-error) 방식으로는 최적 공정 조건을 찾기 매우 어렵다.

본 보고서는 연속체 역학과 비뉴턴 유동 이론을 기반으로 3중 공압출 공정을 체계적으로 분석하며, 특히 PTT 점탄성 모델을 이용한 다이 설계 영향을 정량적으로 다룬다.

\subsection{공정의 물리적 복잡성}

공압출 공정에서 고무 용융체는 다음과 같은 강한 비뉴턴 특성을 보인다:

\begin{enumerate}
  \item \textbf{전단 박화 (Shear Thinning)}: 전단률이 증가하면 점도가 감소하는 현상
  \begin{equation}
  \eta(\dot\gamma) \propto \dot\gamma^{n-1}, \quad 0 < n < 1
  \label{eq:shear_thinning}
  \end{equation}
  
  \item \textbf{정상응력 (Normal Stress)}: 전단 유동에서도 유동 방향 수직으로 응력이 생성
  \begin{equation}
  N_1 = \tau_{11} - \tau_{22}, \quad N_2 = \tau_{22} - \tau_{33}
  \label{eq:normal_stress}
  \end{equation}
  
  \item \textbf{다이 팽윤 (Die Swell)}: 정상응력으로 인한 탄성 회복으로 출구에서 단면 팽창
  
  \item \textbf{점탄성 기억 효과 (Viscoelastic Memory)}: 과거의 변형 이력이 현재 응력에 영향
\end{enumerate}

이러한 현상들을 정확히 모사하려면 단순한 뉴턴 유체 모델로는 부족하며, 고급 점탄성 구성식이 필수적이다.

\section{기계가공과 공정설계: 지배방정식}

\subsection{연속 방정식 (Continuity Equation)}

질량 보존 원리에 따른 연속 방정식은:
\begin{equation}
\frac{D\rho}{Dt} + \rho\nabla \cdot \mathbf{v} = 0
\label{eq:continuity_material}
\end{equation}

비압축 유동($\rho = \text{constant}$)에서는 간단히:
\begin{equation}
\nabla \cdot \mathbf{v}_k = 0; \quad k = I, II, III
\label{eq:continuity_incomp}
\end{equation}

각 층($k = I, II, III$는 크라운, 베이스, 윙)에 대해 독립적으로 질량이 보존되어야 하며, 계면에서도 속도의 법선 성분이 연속이어야 한다.

\subsection{운동량 방정식 (Momentum Equation)}

뉴턴의 제2법칙(관성력 = 작용력)에 따른 운동량 방정식:
\begin{equation}
\rho\frac{D\mathbf{v}}{Dt} = \rho\mathbf{g} + \nabla \cdot \boldsymbol{\sigma}
\label{eq:momentum_general}
\end{equation}

이를 다시 쓰면:
\begin{equation}
\rho\left(\frac{\partial \mathbf{v}}{\partial t} + \mathbf{v} \cdot \nabla\mathbf{v}\right) = \nabla \cdot \boldsymbol{\sigma} + \rho\mathbf{g}
\label{eq:momentum_explicit}
\end{equation}

정상 유동($\partial/\partial t = 0$)에서는:
\begin{equation}
\rho_k(\mathbf{v}_k \cdot \nabla)\mathbf{v}_k = -\nabla p_k + \nabla \cdot \boldsymbol{\tau}_k
\label{eq:momentum_steady}
\end{equation}

여기서:
\begin{itemize}
  \item $\rho_k$: 각 층의 밀도
  \item $\mathbf{v}_k$: 속도 벡터
  \item $p_k$: 정수압(hydrostatic pressure)
  \item $\boldsymbol{\tau}_k$: 편차응력(deviatoric stress)
  \item $\rho\mathbf{g}$: 중력의 영향 (수평 압출에서는 무시)
\end{itemize}

응력 텐서는 등방 응력과 편차응력으로 분해된다:
\begin{equation}
\boldsymbol{\sigma}_k = -p_k\mathbf{I} + \boldsymbol{\tau}_k
\label{eq:stress_decomposition}
\end{equation}

\subsection{편차응력의 물리적 의미}

정수압 $p_k$는 방향성이 없는 등방 압축을 나타내고, 편차응력 $\boldsymbol{\tau}_k$는 방향성 있는 전단 응력과 비등방 응력을 나타낸다. 뉴턴 유체에서는:
\begin{equation}
\boldsymbol{\tau} = 2\eta \mathbf{D}
\end{equation}

하지만 점탄성 유체에서는 과거 이력, 현재 변형률, 회전 효과를 모두 반영해야 하므로 훨씬 복잡한 구성식이 필요하다.

\section{속도 구배의 분해와 객관성 원칙}

\subsection{변형률률 텐서와 회전 텐서}

속도 구배 텐서 $\nabla\mathbf{v}$는 항상 대칭 성분과 반대칭 성분으로 분해된다:
\begin{equation}
\nabla \mathbf{v} = \mathbf{D} + \mathbf{W}
\label{eq:velocity_gradient_decomp}
\end{equation}

변형률률 텐서(Rate-of-Deformation Tensor, 대칭):
\begin{equation}
\mathbf{D} = \frac{1}{2}\left(\nabla\mathbf{v} + (\nabla\mathbf{v})^T\right)
\label{eq:rate_deformation}
\end{equation}

회전 텐서(Spin Tensor, 반대칭):
\begin{equation}
\mathbf{W} = \frac{1}{2}\left(\nabla\mathbf{v} - (\nabla\mathbf{v})^T\right)
\label{eq:spin_tensor}
\end{equation}

물리적 해석:
\begin{itemize}
  \item $\mathbf{D}$: 유체 요소의 모양 변화 속도 - 늘어남, 압축, 전단을 나타냄
  \item $\mathbf{W}$: 유체 요소의 강체 회전 속도 - 좌표계가 회전하는 효과
\end{itemize}

\subsection{객관성 원칙 (Objectivity Principle)}

중요한 물리적 원칙: 순수 강체 회전(pure rigid body motion)에서 재료 내부에 응력이 생기면 안 된다.

순수 회전 상황에서:
\begin{itemize}
  \item $\mathbf{D} = 0$ (모양이 변하지 않음)
  \item $\nabla\mathbf{v} = \mathbf{W} \neq 0$ (회전은 있음)
\end{itemize}

따라서 물리적으로 타당한 구성식은 $\mathbf{D}$와 객관적 시간 미분(objective time derivative)에만 의존해야 한다. 회전 텐서 $\mathbf{W}$는 응력의 직접적인 원인이 될 수 없고, 오직 컨벡티드 미분을 통해 간접적으로만 나타난다.

\subsection{PTT 점탄성 모델}

\section{PTT 점탄성 구성식}

\subsection{편차응력의 분류}

점탄성 유체에서 전체 편차응력 $\boldsymbol{\tau}$는 탄성 성분 $\boldsymbol{\tau}_1$과 점성 성분 $\boldsymbol{\tau}_2$로 분해된다:
\begin{equation}
\boldsymbol{\tau} = \boldsymbol{\tau}_1 + \boldsymbol{\tau}_2
\label{eq:stress_decomp_viscoelastic}
\end{equation}

\begin{itemize}
  \item $\boldsymbol{\tau}_1$: \textbf{탄성 성분(Elastic Component)} - 과거 변형의 이력을 기억하여 정상응력 생성
  \item $\boldsymbol{\tau}_2$: \textbf{점성 성분(Viscous Component)} - 뉴턴 유체처럼 현재 변형률에만 의존
\end{itemize}

\subsection{점성 편차응력}

점성 성분은 뉴턴 유체의 구성식을 따른다:
\begin{equation}
\boldsymbol{\tau}_2 = 2\eta_2 \mathbf{D}
\label{eq:tau_2}
\end{equation}

여기서 $\eta_2$는 점성 점도(viscous viscosity)이고, 이는 현재의 변형률률 $\mathbf{D}$에만 비례한다.

\subsection{탄성 편차응력과 PTT 모델}

파라미터 $\eta_1, \lambda, \varepsilon, \xi$를 가진 PTT(Phan-Thien-Tanner) 구성식:
\begin{equation}
\exp\left[\frac{\varepsilon\lambda}{\eta_1}\mathrm{tr}(\boldsymbol{\tau}_1)\right]\boldsymbol{\tau}_1 + \lambda\left[(1-\tfrac{\xi}{2})\overset{\nabla}{\boldsymbol{\tau}_1} + \tfrac{\xi}{2}\overset{\triangle}{\boldsymbol{\tau}_1}\right] = 2\eta_1\mathbf{D}
\label{eq:ptt}
\end{equation}

여기서:
\begin{itemize}
  \item $\eta_1$: 탄성 점도(elastic viscosity) - 고무 고유의 점도 특성
  \item $\lambda$: 이완시간(relaxation time) - 고분자 사슬의 완화 시간, 초 단위
  \item $\varepsilon$: 신장 거동 파라미터 - 신장응력에 대한 민감도
  \item $\xi$: 전단 거동 파라미터 - 전단 모드의 특성 조절
  \item $\mathrm{tr}(\boldsymbol{\tau}_1)$: 응력 텐서의 자취(trace) - $\tau_{11} + \tau_{22} + \tau_{33}$
\end{itemize}

\subsection{비선형 응력 구성식의 의미}

\subsubsection{지수함수 항의 역할}

\begin{equation}
\exp\left[\frac{\varepsilon\lambda}{\eta_1}\mathrm{tr}(\boldsymbol{\tau}_1)\right]
\label{eq:exp_term}
\end{equation}

이 항은 다음을 담당한다:
\begin{enumerate}
  \item \textbf{신장응력 기억}: 고분자 사슬이 신장될 때 응력이 지수적으로 증가
  \item \textbf{비선형성}: 강한 전단에서는 응력이 비선형으로 변함
  \item \textbf{응력 포화}: 과도한 신장에서 응력 증가가 제한됨 (고분자 사슬의 한계)
\end{enumerate}

\subsubsection{컨벡티드 미분과 회전 효과}

상-컨벡티드 미분(Upper-Convected Derivative):
\begin{equation}
\overset{\triangle}{\boldsymbol{\tau}_1} = \frac{D\boldsymbol{\tau}_1}{Dt} - \boldsymbol{\tau}_1 \cdot (\nabla\mathbf{v})^T - (\nabla\mathbf{v}) \cdot \boldsymbol{\tau}_1
\label{eq:upper_convected_explicit}
\end{equation}

하-컨벡티드 미분(Lower-Convected Derivative):
\begin{equation}
\overset{\nabla}{\boldsymbol{\tau}_1} = \frac{D\boldsymbol{\tau}_1}{Dt} + \boldsymbol{\tau}_1 \cdot (\nabla\mathbf{v})^T + (\nabla\mathbf{v})^T \cdot \boldsymbol{\tau}_1
\label{eq:lower_convected_explicit}
\end{equation}

이 두 미분의 가중 평균:
\begin{equation}
(1-\tfrac{\xi}{2})\overset{\nabla}{\boldsymbol{\tau}_1} + \tfrac{\xi}{2}\overset{\triangle}{\boldsymbol{\tau}_1}
\end{equation}

은 응력의 좌표계 회전에 대한 객관성(objectivity)을 보장하면서도, 매개변수 $\xi$를 통해 특정 고무의 거동에 맞도록 조절한다.

\subsection{PTT 모델의 물리적 해석}

PTT 구성식 (\ref{eq:ptt})의 좌변:
\begin{equation}
\text{(지수함수)} \times \text{(응력)} + \text{(이완시간)} \times \text{(응력의 시간변화)}
\end{equation}

우변:
\begin{equation}
2\eta_1 \mathbf{D} = \text{(점도)} \times \text{(현재 변형률)}
\end{equation}

물리적 의미: 응력의 시간 진화는 현재의 변형률(우변)에 의해 구동되지만, 과거에 축적된 응력(좌변의 지수함수)과 응력의 이완(좌변의 컨벡티드 미분)에 의해 조절된다.

\subsection{핵심 방정식 요약}

PTT 점탄성 구성식:
\begin{equation}
\exp\Big[\frac{\varepsilon \lambda}{\eta_1}\,\mathrm{tr}(\boldsymbol{\tau}_1)\Big]\,\boldsymbol{\tau}_1 + \lambda \Big[(1-\tfrac{\xi}{2})\overset{\nabla}{\boldsymbol{\tau}_1} + \tfrac{\xi}{2}\overset{\triangle}{\boldsymbol{\tau}_1}\Big] = 2\eta_1 \mathbf{D}
\label{eq:ptt}
\end{equation}

지수항만 분리하면:
\begin{equation}
\mathcal{E}(\boldsymbol{\tau}_1) = \exp\Big[\frac{\varepsilon \lambda}{\eta_1}\,\mathrm{tr}(\boldsymbol{\tau}_1)\Big]
\label{eq:exp_term}
\end{equation}

정상응력 차이 정의:
\begin{equation}
N_1 = \tau_{11}-\tau_{22}, \qquad N_2 = \tau_{22}-\tau_{33}
\label{eq:normal_stress}
\end{equation}

\section{다이 팽윤과 유동 특성 분석}

\subsection{다이 팽윤의 물리적 메커니즘}

다이 팽윤(die swell)은 압출 출구에서 고분자 사슬의 탄성 회복과 속도 재분배에 의해 발생하는 현상이다. 발생 과정:

\begin{enumerate}
  \item \textbf{다이 내부 (Pre-exit region)}
  \begin{itemize}
    \item 압축된 유로 내에서 고분자 사슬이 신장됨
    \item 신장응력 축적: 식 (\ref{eq:normal_stress})의 $N_1, N_2$ 증가
    \item 응력 텐서의 자취 $\mathrm{tr}(\boldsymbol{\tau}_1)$이 증가하여 지수함수 항이 강화 (식 \ref{eq:exp_term})
  \end{itemize}
  
  \item \textbf{출구 직후 (Post-exit region)}
  \begin{itemize}
    \item 구속이 갑자기 풀림 → 속도 구배 급변
    \item 축 방향 정상응력 $\tau_{11}$이 급격히 감소
    \item 누적된 탄성 에너지가 방출되면서 횡 방향으로 팽창
  \end{itemize}
  
  \item \textbf{원인이 되는 정상응력 차이}
  \begin{equation}
  N_1 = \tau_{11} - \tau_{22} > 0 \quad \text{(일반적으로 양수)}
  \label{eq:normal_stress_11_22}
  \end{equation}
  
  출구에서 축 방향 제약이 풀리면 축 방향 응력은 빠르게 감소하지만, 횡 방향 응력은 느리게 감소하므로 순 장력이 팽창을 유도한다.
\end{enumerate}

\subsection{다이 팽윤율의 정의}

다이 팽윤율(die swell ratio, swelling index):
\begin{equation}
B = \frac{S - S_0}{S_0}\times 100\% = \frac{\Delta S}{S_0} \times 100\%
\label{eq:swell}
\end{equation}

여기서:
\begin{itemize}
  \item $S_0$: 다이 출구의 단면적
  \item $S$: 압출 후 어느 거리에서 측정한 단면적 (보통 다이 출구로부터 수 cm)
  \item $B$: 백분율로 표현한 팽윤율
\end{itemize}

예를 들어 $B = 20\%$는 출구 단면이 다이 폭의 20% 더 팽창했다는 의미이다.

\subsection{PTT 파라미터와 다이 팽윤의 관계}

같은 다이와 공정 조건에서도 재료에 따라 팽윤이 달라진다. PTT 파라미터의 영향:

\begin{equation}
B = f(\lambda, \eta_1, \varepsilon, \dot\gamma)
\label{eq:swell_ptt}
\end{equation}

\begin{itemize}
  \item \textbf{이완시간 $\lambda$ 증가}: 응력 완화가 느려짐 → 팽윤 증가
  \item \textbf{신장 파라미터 $\varepsilon$ 증가}: 신장응력에 대한 민감도 증가 → 팽윤 증가
  \item \textbf{점도 $\eta_1$ 증가}: 응력 크기 증가 → 팽윤 증가 (전체 흐름 저항 증가)
\end{itemize}

특히 3중 공압출에서는 세 층(크라운, 베이스, 윙)의 $\lambda, \varepsilon$ 값이 서로 다르므로, 각 층의 팽윤율이 상이하다:
\begin{equation}
B_{\text{Crown}} \neq B_{\text{Base}} \neq B_{\text{Wing}}
\label{eq:swell_three_layers}
\end{equation}

이러한 팽윤 차이로 인해 계면 형상이 왜곡되고, 최종 제품의 치수 편차와 형상 불일치가 발생한다.

\section{유동 특성과 질량 보존}

\subsection{유동 균일성 지수}

3중 공압출에서 세 층의 계면 형상은 각 층의 유동 특성에 따라 결정된다. 유동 불균일성(flow non-uniformity):
\begin{equation}
\delta = \frac{v_{\max} - v_{\min}}{v_{\text{avg}}} \times 100\%
\label{eq:flow_uniformity}
\end{equation}

$\delta$가 작을수록:
\begin{itemize}
  \item 출구에서 속도 분포가 균일
  \item 계면 형상이 더 정규적
  \item 제품 두께 편차 감소
  \item 표면 품질 향상
\end{itemize}

\subsection{벽면 전단률}

다이 내부 벽면에서의 전단률:
\begin{equation}
\dot\gamma_w = \left.\frac{dv}{dy}\right|_{\text{wall}}
\label{eq:wall_shear_rate}
\end{equation}

높은 벽면 전단률의 문제점:
\begin{itemize}
  \item 고분자 주쇄의 기계적 절단(shear-induced chain scission)
  \item 재료 분해(thermal degradation) 가속
  \item 표면 결함 발생(표면 주름, 거칠기 증가)
  \item 압출 품질 저하
\end{itemize}

따라서 공정 설계에서는 $\dot\gamma_w$를 허용 범위 이하로 제어해야 한다.

\section{다이 구조 설계 및 공정 파라미터 효과}

\subsection{다이 구조 파라미터}

\subsubsection{다이 너비 (Die Width)}

다이 너비 $W$가 증가할 때:
\begin{equation}
\frac{\partial \delta}{\partial W} < 0, \quad \frac{\partial \dot\gamma_w}{\partial W} < 0
\end{equation}

즉, 너비 증가 → 속도 분산 감소 → 유동 균일성 향상. 동시에 벽면 전단률이 감소하여 표면 결함 위험도 낮아진다. 그러나 다이 너비가 과도하게 증가하면 제조 비용과 다이 구조적 안정성 문제가 발생한다.

\subsubsection{다이 수렴각 (Die Convergence Angle)}

다이 수렴각을 $\alpha$라 할 때:
\begin{equation}
\frac{\partial \dot\gamma_w}{\partial \alpha} > 0, \quad \frac{\partial \delta}{\partial \alpha} \gtrless 0
\end{equation}

수렴각 증가 → 벽면 전단률 급증 → 표면 품질 저하. 동시에 압출 방향에 따라 유동 분배가 급격하게 변하는 고전단 영역이 입구에서 출구로 이동하면서 형상 편차도 증가한다.

\subsection{공정 파라미터}

\subsubsection{유입 유량 (Inlet Flow Rate)}

유입 유량 $Q$와 관련된 특성 전단률 $\dot\gamma_c = Q/A$ (여기서 $A$는 다이 단면적)라 하면:

다이 내부 평균 전단률:
\begin{equation}
\dot\gamma = \frac{Q}{A} \cdot f(\text{geometry})
\end{equation}

유량 증가 → 전단률 증가 → PTT 모델의 비선형성 강화. 식 (\ref{eq:ptt})의 지수함수 항이 강해지면서:
\begin{itemize}
  \item 다이 팽윤 증가 (식 (\ref{eq:swell}))
  \item 벽면 전단률 증가 (표면 결함 위험)
  \item 유동 균일성 저하
\end{itemize}

\subsubsection{인발 속도 (Drawing Speed)}

인발 속도를 $v_d$라 할 때, 다이 입구와 출구 사이의 응력 평형 관계:
\begin{equation}
\frac{\partial B}{\partial v_d} < 0
\end{equation}

즉, 인발 속도 증가 → 다이 팽윤 감소. 이는 출구 이후의 응력 완화 시간이 단축되기 때문이다.

\subsubsection{재료 특성과 전단 효과}

세 개 층의 PTT 파라미터가 상이할 때, 경계층 근처에서 강한 전단이 집중된다:
\begin{equation}
\dot\gamma_{\text{interface}} > \dot\gamma_{\text{bulk}}
\end{equation}

특히 윙팁(wing tip) 영역은 기하학적으로 유로가 좁아 다음이 성립한다:
\begin{equation}
\dot\gamma_w^{\text{wing}} \gg \dot\gamma_w^{\text{bulk}}
\end{equation}

이로 인해 윙팁 영역에 표면 결함(흐름 불안정성, 표면 주름 등)이 선택적으로 발생한다.

\section{다이 설계 최적화 전략}

\subsection{다중 목적 함수 최적화}

공압출 공정의 설계는 상충(conflict)하는 여러 목표를 동시에 만족해야 한다:

\begin{equation}
\min \mathbf{f}(\mathbf{x}) = \begin{bmatrix}
f_1: & \text{형상 편차} & \delta \\
f_2: & \text{벽면 전단률} & \dot\gamma_w \\
f_3: & \text{층간 팽윤 차이} & |B_i - B_j|
\end{bmatrix}
\label{eq:multi_objective}
\end{equation}

설계 변수(design variables):
\begin{equation}
\mathbf{x} = \begin{bmatrix} W & \alpha & Q & v_d \end{bmatrix}^T
\label{eq:design_variables}
\end{equation}

제약조건(constraints):
\begin{equation}
\begin{cases}
\dot\gamma_w \leq \dot\gamma_{\text{crit}} & \text{(표면 결함 방지)} \\
\delta \leq \delta_{\text{max}} & \text{(유동 균일성 요구)} \\
B \leq B_{\text{max}} & \text{(제품 두께 편차 제어)} \\
Q \geq Q_{\text{min}} & \text{(생산성 보장)} \\
v_d \leq v_{\text{max}} & \text{(냉각 능력 고려)}
\end{cases}
\label{eq:constraints}
\end{equation}

\subsection{파라미터 감도 분석}

각 설계 변수가 성능 메트릭에 미치는 영향을 정량화하기 위해 감도 분석(sensitivity analysis)을 수행한다:

\begin{equation}
S_{ij} = \frac{\partial f_i}{\partial x_j} \bigg|_{\mathbf{x}_0}
\label{eq:sensitivity}
\end{equation}

예를 들어:

\begin{itemize}
  \item \textbf{다이 너비 W에 대한 유동 균일성 감도}:
  \begin{equation}
  S_{\delta,W} = \frac{\partial \delta}{\partial W} < 0 \quad \text{(너비 증가 → 균일성 향상)}
  \end{equation}
  
  \item \textbf{수렴각 $\alpha$에 대한 벽면 전단률 감도}:
  \begin{equation}
  S_{\dot\gamma_w,\alpha} = \frac{\partial \dot\gamma_w}{\partial \alpha} > 0 \quad \text{(수렴각 증가 → 전단률 증가)}
  \end{equation}
  
  \item \textbf{유량 Q에 대한 다이 팽윤 감도}:
  \begin{equation}
  S_{B,Q} = \frac{\partial B}{\partial Q} > 0 \quad \text{(유량 증가 → 팽윤 증가)}
  \end{equation}
\end{itemize}

\subsection{파렛토 최적해(Pareto Optimal Solution)}

다중 목적 최적화에서는 여러 파렛토 최적해가 존재한다. 이는 한 목표를 개선하려면 다른 목표를 악화시켜야 하는 상황을 의미한다:

\begin{equation}
\text{파렛토 최적해} \Leftrightarrow \neg \exists \mathbf{x}' \in \mathcal{F}: f_i(\mathbf{x}') \leq f_i(\mathbf{x}) \, \forall i
\label{eq:pareto_optimality}
\end{equation}

여기서 $\mathcal{F}$는 가능(feasible) 영역이다.

실제 공정 선택은 가중 스칼라화(weighted scalarization)를 통해 단일 목적함수로 변환한다:

\begin{equation}
\min \sum_{i=1}^3 w_i f_i(\mathbf{x}), \quad \sum w_i = 1, \, w_i \geq 0
\label{eq:weighted_sum}
\end{equation}

\section{3중 공압출의 계면 동역학}

\subsection{계면 형상 결정}

세 층의 경계면 위치는 다음 원칙으로 결정된다:

\begin{enumerate}
  \item \textbf{질량 보존}: 각 층의 체적 유량 연속성
  \begin{equation}
  Q_k = \int_{\Omega_k} v_n(\mathbf{r}) \, dA, \quad k = I, II, III
  \label{eq:volume_conservation}
  \end{equation}
  
  \item \textbf{응력 평형}: 계면에서 응력 연속성
  \begin{equation}
  \boldsymbol{\tau}_k^{\text{left}} \cdot \mathbf{n} = \boldsymbol{\tau}_{k+1}^{\text{right}} \cdot \mathbf{n}
  \label{eq:stress_continuity}
  \end{equation}
  
  \item \textbf{속도 연속성}: 법선 방향 속도가 같음
  \begin{equation}
  v_{n,k} = v_{n,k+1} \quad \text{at interface}
  \label{eq:velocity_continuity}
  \end{equation}
\end{enumerate}

\subsection{층간 상호 작용}

각 층의 PTT 파라미터가 다르면, 유동 특성에 큰 차이가 난다:

\begin{equation*}
\begin{aligned}
\text{Crown}: &\quad \lambda_C, \varepsilon_C \; \Rightarrow\; B_C = f(\lambda_C, \varepsilon_C, \dot\gamma_C) \\
\text{Base}: &\quad \lambda_B, \varepsilon_B \; \Rightarrow\; B_B = f(\lambda_B, \varepsilon_B, \dot\gamma_B) \\
\text{Wing}: &\quad \lambda_W, \varepsilon_W \; \Rightarrow\; B_W = f(\lambda_W, \varepsilon_W, \dot\gamma_W)
\end{aligned}
\end{equation*}

일반적으로 $B_C > B_B > B_W$이므로, 계면이 곡선 형태로 왜곡된다:

\begin{equation}
\text{계면 곡률} \propto |B_i - B_j| \quad \Rightarrow \text{계면 형상 왜곡}
\label{eq:interface_curvature}
\end{equation}

\section{수치해석 방법론 개요}

3중 공압출 PTT 점탄성 유동 해석은 식 (\ref{eq:ptt})이 포함된 강한 비선형 편미분방정식 시스템이다. 본 보고서에서는 상용 CFD 솔버인 ANSYS Polyflow 기준으로 다음 절차로 해석한다고 기술한다.

\begin{enumerate}
  \item \textbf{격자}: 다이 입출구/벽면 근처 자동 세분화, 계면 추적을 위한 적응 메시
  \item \textbf{물성}: 각 층의 PTT 파라미터 $(\eta_1,\lambda,\varepsilon,\xi)$ 입력, 밀도/점도/온도 조건 설정
  \item \textbf{경계 조건}: 입구 유량 또는 압력, 출구 압력 또는 인발 속도, 벽면 무미끄럼
  \item \textbf{해석}: 안정화된 FEM + 뉴턴-랍슨 반복으로 수렴, 필요 시 안정화 옵션 활용
  \item \textbf{결과}: 다이 팽윤, 층간 계면 형상, 속도/전단률/정상응력 분포 후처리
\end{enumerate}

계산 시간은 격자 규모(약 5만~30만 요소)와 수렴 조건에 따라 수 시간~수십 시간이 소요될 수 있다.

  이를 컴팩트하게 쓰면 다음과 같은 비선형 지배방정식 시스템으로 표현된다:

  \begin{equation}
  \mathcal{G}(\mathbf{v}, p, \boldsymbol{\tau}_1) = 
  \begin{cases}
  \nabla \cdot \mathbf{v} = 0 \\[4pt]
  \rho (\mathbf{v} \cdot \nabla) \mathbf{v} = -\nabla p + \nabla \cdot (2\eta_0 \mathbf{D} + \boldsymbol{\tau}_1) \\[4pt]
  \exp\Big[\frac{\varepsilon \lambda}{\eta_1} \mathrm{tr}(\boldsymbol{\tau}_1)\Big] \boldsymbol{\tau}_1 + \lambda \Big[(1-\tfrac{\xi}{2})\overset{\nabla}{\boldsymbol{\tau}_1} + \tfrac{\xi}{2}\overset{\triangle}{\boldsymbol{\tau}_1}\Big] = 2\eta_1 \mathbf{D}
  \end{cases}
  \label{eq:governing_full}
  \end{equation}

\section{실제 공정 설계 사례}

\subsection{최적화 문제의 실제 적용}

다이 너비, 수렴각, 유량, 인발 속도를 설계 변수로 하는 다중 목적 최적화:

\begin{equation}
\begin{array}{ll}
\min & J = w_1 \delta_{\text{outlet}} + w_2 \max(\dot\gamma_w) + w_3 \max(|B_i - B_j|) + w_4 Q_{\text{cost}} \\
\text{subject to} & Q_{\min} \leq Q \leq Q_{\max} \\
& \alpha_{\min} \leq \alpha \leq \alpha_{\max} \\
& W_{\min} \leq W \leq W_{\max} \\
& v_d \leq v_{d,\max}
\end{array}
\label{eq:optimization_practical}
\end{equation}

\subsubsection{설계 가중치의 설정}

\begin{itemize}
  \item $w_1 = 0.3$: 형상 편차 (정밀도 우선)
  \item $w_2 = 0.4$: 벽면 전단률 (표면 품질 우선)
  \item $w_3 = 0.2$: 층간 팽윤 차이 (계면 형상 제어)
  \item $w_4 = 0.1$: 생산 비용 (경제성)
\end{itemize}

\section{결론 및 설계 지침}

\subsection{주요 결과}

타이어 트레드 3중 공압출 공정에 대한 본 연구의 주요 성과:

\begin{enumerate}
  \item \textbf{기하학적 정식화}: 식 (\ref{eq:governing_full})의 PTT 점탄성 모델을 통해 타이어 트레드 3중 공압출 유동을 정확히 모사할 수 있음을 보였다.
  
  \item \textbf{다이 팽윤 메커니즘}: 식 (\ref{eq:swell})과 (\ref{eq:ptt})의 관계로부터, 다이 팽윤이 응력 텐서의 자취 $\mathrm{tr}(\boldsymbol{\tau}_1)$과 정상응력 $N_1$에 의존함을 명시적으로 도출했다.
  
  \item \textbf{계면 왜곡 원인}: 식 (\ref{eq:interface_curvature})를 통해 층별 팽윤 차이 $|B_i - B_j|$가 계면 형상 왜곡의 직접적 원인임을 보였다.
  
  \item \textbf{다이 구조 영향}: 식 (\ref{eq:die_width_effects})와 (\ref{eq:convergence_angle_effects})로부터, 다이 너비와 수렴각이 유동 균일성과 벽면 전단률에 미치는 정량적 영향을 정식화했다.
  
  \item \textbf{공정 최적화 전략}: 다중 목적 최적화 제약 조건 하에서 파렛토 최적해를 도출하는 방법론을 제시했다.
\end{enumerate}

\subsection{공정 설계 지침}

\subsubsection{다이 너비 선택}

다이 너비가 커질수록 출구 형상 편차가 완만히 감소한다는 경험식을 사용한다:

\begin{equation}
\delta(W) \approx \delta_0 \left(\frac{W_0}{W}\right)^{n_\delta}, \quad n_\delta \in [0.6, 0.9]
\label{eq:die_width_effects}
\end{equation}

\begin{equation}
W^* = \arg\min_W \left(\delta(W) + \lambda_{\text{cost}} C(W)\right)
\label{eq:optimal_width}
\end{equation}

권장사항:
\begin{itemize}
  \item 형상 정밀도 우선: $W$를 크게 설정 ($\geq 100$ mm)
  \item 표면 품질 중시: $W$를 중간 범위 유지 (50-80 mm)
  \item 비용 우선: $W$를 작게 설정 ($\leq 50$ mm) - 대신 다이 내부 피드 블록 설계 최적화
\end{itemize}

\subsubsection{수렴각 선택}

수렴각이 커지면 벽면 전단률이 거의 선형적으로 증가한다:

\begin{equation}
\dot\gamma_w(\alpha) \approx \dot\gamma_{w,0} \left(1 + k_\alpha \tan \alpha\right), \quad k_\alpha \approx 0.8{-}1.2
\label{eq:convergence_angle_effects}
\end{equation}

벽면 전단률 제약:
\begin{equation}
\alpha_{\max} = \arctan\left(\frac{\dot\gamma_{\text{crit}} \cdot h}{v_{\text{exit}}}\right)
\label{eq:max_convergence}
\end{equation}

권장사항:
\begin{itemize}
  \item 표면 품질 우선: $\alpha = 5°$-$10°$ (완만한 수렴, 낮은 전단)
  \item 균형 설계: $\alpha = 10°$-$15°$ (적당한 전단, 합리적 다이 길이)
  \item 비용 최적화: $\alpha = 15°$-$20°$ (높은 전단, 짧은 다이) - 재료 제어 필수
\end{itemize}

\subsubsection{유입 유량 설정}

유량을 높이면 다이 팽윤이 완만히 증가한다는 근사식을 사용한다:

\begin{equation}
B(Q) \approx B_0 \left[1 + k_Q \left(\frac{Q}{Q_0} - 1\right)\right], \quad k_Q \sim 0.3{-}0.6
\label{eq:flow_rate_effects}
\end{equation}

이를 이용해 목표 형상에 맞춘 보정식을 정의한다 (식 \ref{eq:flow_rate_effects} 기반):

\begin{equation}
Q_{\text{opt}} = Q_{\text{nominal}} \times \left(1 - \frac{\delta_{\text{target}} - \delta_{\text{actual}}}{\delta_{\text{max}}}\right)
\label{eq:flow_tuning}
\end{equation}

권장사항:
\begin{itemize}
  \item 생산 목표량 > 형상 정밀도: $Q = Q_{\text{max}}$ (고속 생산, 정밀도 보정)
  \item 형상 정밀도 > 생산량: $Q = 0.7 \times Q_{\text{max}}$ (저속 운전, 안정성 우선)
  \item 중간 타협: $Q = 0.85 \times Q_{\text{max}}$ (생산성과 품질 병행)
\end{itemize}

\subsubsection{인발 속도 최적화}

인발 속도가 증가하면 외부 팽윤이 감소한다는 실험적 추세를 반영한다:

\begin{equation}
B(v_d) \approx B_{\text{free}} - k_v \left(\frac{v_d}{v_{\text{ref}}} - 1\right), \quad k_v \sim 0.4{-}0.7
\label{eq:draw_speed_swell}
\end{equation}

식 (\ref{eq:draw_speed_swell})의 역의 관계 활용:

\begin{equation}
v_d = \frac{Q}{S_0} \cdot \kappa, \quad \kappa = 1 + \frac{B_{\text{target}} - B_{\text{actual}}}{B_{\text{max}}}
\label{eq:draw_speed_opt}
\end{equation}

권장사항:
\begin{itemize}
  \item 다이 팽윤 과다 ($B > 15\%$): $v_d$ 증가 (인발 속도 2-3\% 상향)
  \item 다이 팽윤 부족 ($B < 5\%$): $v_d$ 감소 (인발 속도 2-3\% 하향)
  \item 적정 범위 ($5\% < B < 15\%$): $v_d$ 유지
\end{itemize}

\subsection{고급 제어 전략}

\subsubsection{피드 블록 설계}

각 층의 유량을 독립적으로 제어하기 위해 피드 블록 설계:

\begin{equation}
h_k^{\text{feed}} = \frac{Q_k \cdot L}{v_{\text{nominal}} \cdot W}
\label{eq:feed_block_height}
\end{equation}

여기서 $k = \text{Crown}, \text{Base}, \text{Wing}$이다.

\subsubsection{실시간 모니터링}

\begin{itemize}
  \item \textbf{압력 센서}: 다이 입구, 출구 → 흐름 저항 및 팽윤 추정
  \item \textbf{온도 센서}: 다이 내부 → 점도 변화 감지
  \item \textbf{카메라}: 압출 출구 → 형상 편차 실시간 모니터링
  \item \textbf{인발 힘 센서}: 제품 인발력 → 응력 상태 추정
\end{itemize}

피드백 제어:
\begin{equation}
\Delta Q = K_P (B_{\text{target}} - B_{\text{actual}}) + K_I \int (B_{\text{target}} - B_{\text{actual}}) \, dt
\label{eq:feedback_control}
\end{equation}

\subsection{향후 연구 과제}

\begin{enumerate}
  \item \textbf{3D 수치해석}: 현재는 2D 또는 축대칭 해석이 주류 → 완전 3D 다이 형상에서의 이방성 효과 고려
  \item \textbf{열 이송}: 점도의 온도 의존성, 전단 가열, 냉각 효과 포함
  \item \textbf{비뉴턴 성질}: PTT 모델을 넘어 더 고급 모델(multi-mode 모델, 비선형 비탄성) 적용
  \item \textbf{표면 불안정성}: 전단 화이트닝(shear whitening), 표면 결함의 근본 원인 규명
  \item \textbf{실험 검증}: 수치해석 결과와 실제 압출 실험의 체계적 비교
\end{enumerate}

\subsection{최종 결론}

본 보고서는 연속체 역학의 기본 원리(연속 방정식, 운동량 방정식), 점탄성 구성식(PTT 모델 \ref{eq:ptt}), 수치해석 기법(유한 요소법, 뉴턴-랍슨 반복)을 종합적으로 적용하여, 타이어 트레드 3중 공압출 공정의 설계와 최적화에 대한 이론적 기초와 실무적 지침을 제시했다.

특히 파워포인트에서 제시된 핵심 식들:
\begin{itemize}
  \item 연속 방정식: $\nabla \cdot \mathbf{v}_k = 0$
  \item 운동량 방정식: $\rho_k(\mathbf{v}_k \cdot \nabla)\mathbf{v}_k = -\nabla p_k + \nabla \cdot \boldsymbol{\tau}_k$
  \item PTT 구성식: $\exp[\frac{\varepsilon\lambda}{\eta_1}\mathrm{tr}(\boldsymbol{\tau}_1)]\boldsymbol{\tau}_1 + \lambda\left[(1-\tfrac{\xi}{2})\overset{\nabla}{\boldsymbol{\tau}_1} + \tfrac{\xi}{2}\overset{\triangle}{\boldsymbol{\tau}_1}\right] = 2\eta_1\mathbf{D}$
\end{itemize}

을 통해 고무 용융체의 비뉴턴, 점탄성 유동 거동과 다이 팽윤, 계면 왜곡, 표면 결함의 근본적 메커니즘을 밝혔다.

이러한 이론적 이해를 바탕으로, 공정 설계자는 시행착오 없이도 최적의 다이 구조와 공정 조건을 체계적으로 도출할 수 있으며, 궁극적으로 고품질 타이어 제품의 효율적 생산을 실현할 수 있다.

\section*{참고문헌}
(본 보고서 초안에서는 참고문헌 생략)

\end{document}
