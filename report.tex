% report.tex - 간단한 한글 LaTeX 보고서 템플릿 (XeLaTeX 사용 권장)
\documentclass[10pt,a4paper]{article}

% 한글 설정: fontspec + xeCJK
% 라틴: TeX Live 기본 Latin Modern, 한글: Noto Sans CJK KR (fonts-noto-cjk)
\usepackage{fontspec}
\usepackage{xeCJK}
\defaultfontfeatures{Ligatures=TeX,Scale=MatchLowercase}
\setmainfont{Latin Modern Roman}
\setsansfont{Latin Modern Sans}
\setmonofont{Latin Modern Mono}
\setCJKmainfont{Noto Sans CJK KR}
\setCJKsansfont{Noto Sans CJK KR}
% CJK 띄어쓰기 처리
\xeCJKsetup{CJKspace=true, CJKglue=\hskip 0pt plus 0.08\baselineskip}
\xeCJKsetcharclass{"AC00}{"D7A3}{1}

\usepackage{amsmath,amssymb}
\usepackage{graphicx}
\usepackage{caption}
\usepackage{booktabs}
\usepackage{hyperref}
\hypersetup{colorlinks=true, linkcolor=blue, urlcolor=blue}

% 문서 정보
\title{타이어 공정설계 보고서}
\author{
  지명훈 (2020074248) \\
  지종호 (2021004939) \\
  권준영 (2021070719)
}
\date{\today}

\begin{document}
\maketitle
\begin{abstract}
본 보고서는 타이어를 다층 복합재 관점에서 바라보고, 정련 $\to$ 반제품(압출·캘린더링·비드) $\to$ 성형 $\to$ 가류 $\to$ 검사로 이어지는 공정 체인을 품질지표(특히 균일성)와 직접 연결해 해석한다. 트레드 \textbf{3중 공압출(캡/베이스/윙)}에 대해 연속체 역학과 PTT 점탄성 구성식을 적용하여 다이 팽윤·벽면 전단률·계면 변형의 발생 메커니즘을 기술하고, \textbf{다이 너비} ($W$), \textbf{수렴각} ($\alpha$), \textbf{유량} ($Q$), \textbf{인발속도} ($v_d$)가 \textbf{형상 편차} ($\delta$), \textbf{전단률} ($\dot{\gamma}_w$), \textbf{층간 팽윤차} ($|B_i-B_j|$)에 미치는 감도를 정리하였다. 또한 Arrhenius 기반 가류 반응과 \textbf{열전달} ($\rho c \, \partial T/\partial t = k \nabla^{2} T$)을 결합하여 미·과가류의 정량 기준과 온도·압력 연동(PT-coupled) 프로파일의 운영 원칙을 제시했다. 결과적으로, 공압출과 가류 파라미터는 유니포미티·표면품질·내구로 수렴되는 공정-성능 인과를 형성하며, 이를 다중목적 최적화 프레임으로 의사결정(정밀도/생산성/비용)을 지원할 수 있음을 보였다. 본 연구는 경향성 중심의 모델 한계와 현장 데이터 부족을 인정하며, 고급 구성식·열-유동 연성·엔드투엔드 통합 시뮬레이션 및 공정능력지수 ($C_p/C_{pk}$) 기반 검증을 후속 과제로 제안한다.
\end{abstract}

\tableofcontents
\newpage

\section{서론}

\subsection{타이어 공정의 전체 흐름}

타이어를 다층 복합재로 보고, 정련·압출·압연·반제품·성형·가류·검사에 이르는 전체 공정을 계통적으로 정리할 수 있다. 본 보고서에서는 이러한 공정 체인 중에서 공정 능력 관점에서 트레드 공압출과 가류 공정을 중점적으로 분석하였다.

\subsection{타이어의 기능과 성과지표}

\subsubsection{타이어의 핵심 기능}

타이어는 차량과 노면 사이의 유일한 접점으로서 다음의 기능을 수행한다:

\begin{itemize}
  \item \textbf{차량 하중 지지}: 차량과 승객, 화물의 전체 중량을 안전하게 지탱
  \item \textbf{가속력과 제동력 전달}: 엔진의 구동력과 브레이크의 제동력을 지면에 효율적으로 전달
  \item \textbf{조향}: 운전자의 조향 입력을 노면에 전달하여 차량 방향 제어
  \item \textbf{지면 충격 흡수}: 노면의 요철과 충격을 흡수하여 승차감 향상
\end{itemize}

\subsubsection{타이어 성과지표(KPI)}

타이어 시장에서 중요시하는 핵심 성과지표는 다음과 같다:

\begin{enumerate}
  \item \textbf{연비(Fuel Efficiency)}: 구름 저항 최소화를 통한 연료 소비 절감
  \item \textbf{수명(Durability)}: 타이어의 내마모성과 전체 주행 가능 거리
  \item \textbf{젖은 노면 접지력(Wet Grip)}: 비오는 도로에서의 제동 성능과 안전성
  \item \textbf{마른 노면 접지력(Dry Grip)}: 일반 도로에서의 가속·제동·코너링 성능
  \item \textbf{조향 안정성(Handling)}: 코너링 반응성과 주행 안정성
  \item \textbf{승차감/정숙성(Comfort/Noise)}: 진동 흡수와 소음 저감
\end{enumerate}

\subsection{타이어 균일성과 품질 지표}

\noindent \textit{\color{blue}보고서 진행 상황: \href{https://roids7.github.io/tire-report/tire_report.html}{프로젝트 진행 상황 페이지}}를 참고하여 각 단계별 완료 현황을 확인할 수 있습니다.

타이어의 균일성은 제품 품질과 안전성에 직접적인 영향을 미치는 핵심 지표이다. 본 보고서는 시뮬레이션과 데이터 기반 제어 개념을 활용한 개선 가능성을 살펴보았다. 

흐름을 단계별로 배열함으로써, 개별 공정이 아닌 공정 체인으로 이해할 수 있게 되었다. 특히 공정 파라미터가 균일성 향상이라는 품질 지표를 유동 현상과 직접 연결할 수 있었으며, 기계가공과 공정설계 과목에서 다루는 공정 능력·변동 관리 개념을 실제 산업 공정에 접목할 수 있었다.

\subsection{배경 및 연구 목표}

승용차 타이어 트레드는 세 가지 고무 컴파운드(크라운 Crown, 베이스 Base, 윙 Wing)를 동시에 성형하는 3중 공압출(tri-composite co-extrusion) 기술로 제조된다. 이 공정은 다음과 같은 특징을 가진다:

\begin{itemize}
  \item \textbf{크라운(Crown)}: 노면과 직접 접촉하여 마모 저항성, 그립감, 미끄럼 방지를 담당
  \item \textbf{베이스(Base)}: 구조적 강도, 발열 특성, 내구성을 담당하는 중간층
  \item \textbf{윙(Wing)}: 코너링 안정성, 소음 감소, 승차감 미세 조정을 담당
\end{itemize}

고무 용융체의 강한 점탄성 거동과 다이 구조의 복잡성으로 인해 유동 분포, 형상, 표면 품질이 복잡하게 결정된다. 특히 세 층 고무의 서로 다른 점탄성 특성(점도, 이완시간, 신장 민감도)으로 인해 전통적인 시험-수정(trial-and-error) 방식으로는 최적 공정 조건을 찾기 매우 어렵다.

\subsection{공정의 물리적 복잡성}

공압출 공정에서 고무 용융체는 다음과 같은 강한 비뉴턴 특성을 보인다:

\begin{enumerate}
  \item \textbf{전단 박화 (Shear Thinning)}: 전단률이 증가하면 점도가 감소하는 현상
  \begin{equation}
  \eta(\dot\gamma) \propto \dot\gamma^{n-1}, \quad 0 < n < 1
  \label{eq:shear_thinning}
  \end{equation}
  
  \item \textbf{정상응력 (Normal Stress)}: 전단 유동에서도 유동 방향 수직으로 응력이 생성
  \begin{equation}
  N_1 = \tau_{11} - \tau_{22}, \quad N_2 = \tau_{22} - \tau_{33}
  \label{eq:normal_stress}
  \end{equation}
  
  \item \textbf{다이 팽윤 (Die Swell)}: 정상응력으로 인한 탄성 회복으로 출구에서 단면 팽창
  
  \item \textbf{점탄성 기억 효과 (Viscoelastic Memory)}: 과거의 변형 이력이 현재 응력에 영향
\end{enumerate}

이러한 현상들을 정확히 모사하려면 단순한 뉴턴 유체 모델로는 부족하며, 고급 점탄성 구성식이 필수적이다.

\section{타이어 구조}

\subsection{다층 복합재로서의 타이어}

타이어는 단순 고무가 아닌 다층 복합재로 구성되며, 각 층이 서로 다른 기능을 담당한다. 주요 구성 요소는 다음과 같다.

\subsection{이너라이너(Inner Liner)}

\textbf{역할}: 공기 차단으로 타이어 내압 유지.

\textbf{소재}: BIIR/CIIR 저투과성 고무(Butyl/Chlorobutyl Rubber).

\textbf{공정}: PCR(승용차용)은 압출 비중이 높으며, TBR(트럭/버스용)은 캘린더 시트도 흔함.

\textbf{포인트}: 핀홀·스플라이스 누기 방지, 게이지 균일성 확보.

\subsection{카카스 플라이(Carcass Ply)}

\textbf{역할}: 타이어 기본 골격으로, 비드에서 비드까지 연속적으로 구성.

\textbf{소재}: 폴리에스터/나일론 섬유 코드.

\textbf{공정}: 코드를 RFL(Resorcinol-Formaldehyde-Latex) 딥 처리 후 캘린더로 양면 스킴 코팅, 바이어스 각도로 절단.

\textbf{포인트}: 코드 각도·EPI(Ends Per Inch) 균일성, 턴업(비드 감싸기) 접착 강도.

\subsection{스틸 코드 벨트(Steel Belt Package)}

\textbf{역할}: 크라운 강성 증가, 원주 성장 억제, 트레드 접지 안정성 향상.

\textbf{소재}: 브라스 도금 스틸 코드(보통 2--3겹, $\pm(15-25)°$ 배치).

\textbf{공정}: 캘린더 양면 스킴 $\to$ 바이어스 각도 절단 $\to$ 성형 시 정합 적층.

\textbf{포인트}: 각도/폭 정합, 엣지 피로·박리 방지(스텝 최소화).

\subsection{와인딩 밴디지(캡플라이/스파이럴 오버레이)}

\textbf{역할}: 고속에서 벨트 에지 구속, 원주 성장 억제, 유니포미티 보조.

\textbf{소재}: 나일론/아라미드 코드 슬릿 테이프.

\textbf{공정}: 연속 스파이럴 와인딩 또는 조인트 캡플라이 적층.

\textbf{포인트}: 장력·피치 균일성, 겹침 단차 최소화.

\subsection{트레드(Tread)}

\textbf{역할}: 마찰·제동·마모·연비 성능 담당.

\textbf{소재}: 캡(그립 중시)/베이스(발열 저감) 등 다층 구조.

\textbf{공정}: 공압출(듀플렉스/트리플렉스) + 프로파일 다이 + 인라인 게이지 측정.

\textbf{포인트}: 층간 접착, 프로파일·센터링, 스코치 관리.

\subsection{러버 사이드월(Sidewall)}

\textbf{역할}: 케이싱 보호, 내오존·내충격성, 연석 스커핑(연석 마찰로 인한 외측 마모·긁힘) 방어.

\textbf{소재}: 내오존·내굴곡 배합(필요 시 컬러/화이트 레터).

\textbf{공정}: 압출(스트립) $\to$ 성형 시 스티칭.

\textbf{포인트}: 표면 미세 결함 방지(출구 전단·온도 관리), 두께 균일성.

\subsection{비드(Bead) 어셈블리}

비드는 여러 구성요소로 이루어진 복합 구조이다:

\subsubsection{스틸 코어(Bead Core)}

\textbf{역할}: 림 체결 및 기밀 유지.

\textbf{소재/공정}: 고탄소강 와이어 권선·성형(브라스 코팅).

\textbf{포인트}: 동심도·외경 공차 관리.

\subsubsection{에이펙스(Apex)}

\textbf{역할}: 비드 위 삼각 쐐기 형태로 하부 강성 경사 형성(승차감·조종성·내구성 조율).

\textbf{소재/공정}: 고경도 고무 압출 $\to$ 비드에 열착.

\textbf{포인트}: 각도·높이·질량 관리.

\subsubsection{플리퍼(Flipper)}

\textbf{역할}: 비드–플라이 사이 직물 절연층, 마모·컷 방지.

\textbf{공정}: 코팅 패브릭 적층.

\subsubsection{치퍼 와이어(Chipper Wire)}

\textbf{역할}: 플랜지·포트홀 충격에서 플라이 보호(하부 사이드월).

\textbf{공정}: 강선 링/스트립 적층(모델에 따라).

\subsubsection{채퍼(Chafer)}

\textbf{역할}: 림 접촉부 경질 고무 스트립으로 마찰·마모 방어, 림 고정 보조.

\textbf{공정}: 압출 스트립 적층.

\subsubsection{림가드(Rim Guard, 옵션)}

\textbf{역할}: 림 손상·긁힘 방지용 돌출부(주로 UHP/SUV 타이어).

\textbf{공정}: 트레드/사이드월과 공압출 또는 별도 압출.

\subsection{공정과 최종 물성의 관계}

배합 설계, 공압출, 가류 등 각 공정이 순차적으로 작용하여 최종 타이어 물성이 결정된다.

\section{타이어 제조 공정 개요}

타이어 제조는 배합(Compounding) $\to$ 반제품 제조(Semi-finished Products) $\to$ 성형(Building) $\to$ 가류(Curing) $\to$ 검사(Inspection)의 단계로 구성된다.

\subsection{타이어 정련(제련, Compounding)}

고무, 카본블랙, 실리카, 가소제, 가황제 등을 배합하여 최종 성능에 맞는 컴파운드를 제조하는 공정이다. 이 단계에서 트레드, 사이드월, 이너라이너 등 각 부위별 고무 배합물이 준비된다.

\subsection{반제품 제조 공정}

\subsubsection{압출(Extrusion) — 트레드·사이드월·에이펙스 등}

혼련된 고무를 압출기에 투입해 트레드, 사이드월, 에이펙스(비드 필러) 등 형상 부품을 연속 성형한다. 트레드는 보통 캡/베이스(±윙)를 동시에 뽑는 공압출을 사용하며, 다이 통과 후 냉각·길이계측·절단을 거쳐 성형공정 투입용 길이로 준비한다. 

\textbf{이 단계의 핵심}은 프로파일(폭·두께·홈 깊이)과 층간 접착의 안정화, 그리고 스플라이스 길이·센터링을 성형 기준에 맞추는 것이다. 표면 결함이나 치수 산포는 가류 후 교정이 어렵기 때문에 인라인 게이지/비전으로 즉시 보정하는 것이 일반적이다.

\textbf{주요 불량 및 대응}:
\begin{itemize}
  \item \textbf{스웰·치수 산포} → 다이·냉각 동시 최적화
  \item \textbf{계면 주름·분리} → 유로 밸런스·온도/압력 매칭
  \item \textbf{표면 결함} → 출구 전단·온도 낮춤
\end{itemize}

\subsubsection{압연/캘린더링(Calendering) — 플라이·벨트·시트}

섬유(PE/PA 등)·스틸 코드에 양면 스킴 고무를 코팅하여 카카스 플라이, 스틸 벨트, 캡플라이/오버레이 시트를 만들고, 필요 시 이너라이너 시트도 생산한다. 캘린더 롤 니프에서 목표 게이지를 형성한 뒤 냉각·계측·권취까지 일괄 수행하며, 제품에 따라 절단 각(바이어스 각)을 부여해 적층 준비를 마친다.

\textbf{공정의 포인트}는 폭 방향 두께 균일, 코드 각도·밀도(EPI)의 안정, 그리고 브라스–고무(RFL 포함) 계면의 접착 품질이다. 정지·재기동에 따른 열형상이 게이지에 직접 영향을 주므로 연속 운전과 온도 램프 관리가 중요하다.

\textbf{주요 불량 및 대응}:
\begin{itemize}
  \item \textbf{엣지헤비/센터라이트} → 크로스액시스·롤벤딩·연삭 재검
  \item \textbf{스팟 게이지(정지흔적)} → 연속운전·온도램프
  \item \textbf{코드노출·편주입} → 가이드·텐션·니프 재세팅
\end{itemize}

\subsubsection{비드 공정(Bead Assembly) — 링/필러 모듈}

브라스 코팅 스틸 와이어를 규정 회수로 권선하여 비드 링을 만들고, 여기에 \textbf{에이펙스(고경도 삼각 웨지)}와 채퍼/플리퍼/치퍼 등 보강 부품을 적층해 모듈화한다. 비드는 림과의 기계적 결합과 기밀 유지의 기초이므로 동심도·외경 공차를 엄격히 관리한다.

\textbf{에이펙스의 역할}: 에이펙스의 각도·높이·질량은 하부 사이드월의 강성 구배를 형성해 승차감·조종성·내구에 영향을 준다. 스티칭·열착 조건을 통해 박리나 헐거움을 방지한다.

\textbf{성형 기준점}: 이 모듈은 이후 성형 드럼에서 카카스 턴업의 기준점으로 작동한다.

\textbf{주요 불량 및 대응}:
\begin{itemize}
  \item \textbf{타원/면정렬 불량} → 성형·권선 텐션 보정
  \item \textbf{에이펙스 헐거움·박리} → 온도·압력·스티치 재설정
\end{itemize}

\subsection{성형 공정(Tire Building Process)}

타이어 성형공정은 각 부품을 조립하여 원형의 타이어 형태를 만드는 과정이다. 이 공정에서는 고무와 보강재로 이루어진 여러 층의 반제품을 일정 순서에 따라 적층하여, '그린 타이어(Green Tire)'라고 불리는 미가황 타이어를 제작한다. 성형공정은 타이어의 구조적 완성도와 품질에 직접적인 영향을 미치는 핵심 단계이다.

\subsubsection{성형장비 구성}

타이어 성형은 일반적으로 \textbf{타이어 빌딩 머신(Tire Building Machine)}을 사용하여 진행된다. 이 장비는 타이어의 각 부품을 정해진 위치에 순차적으로 적층하도록 제어되며, 주로 드럼(drum), 피더(feeder), 적층장치(applicator) 등으로 구성된다. 드럼은 타이어의 원통형 형상을 형성하는 역할을 하며, 성형 중 확장과 수축이 가능하다.

\subsubsection{성형 단계}

\begin{enumerate}
  \item \textbf{내부 라이너(Inner Liner) 장착}: 성형의 첫 단계로 내부 라이너를 드럼 위에 장착한다. 내부 라이너는 공기 누출을 방지하는 역할을 하며, 주로 불투과성이 높은 합성고무가 사용된다.
  
  \item \textbf{카카스 플라이(Carcass Ply) 적층}: 카카스 플라이는 타이어의 기본 골격을 이루는 섬유층으로, 주로 폴리에스터, 나일론 등의 섬유 코드가 사용된다. 이 층은 비드를 감싸면서 드럼에 적층되어 타이어의 형태를 형성한다.
  
  \item \textbf{비드 와이어(Bead Wire) 삽입}: 다음으로 비드 와이어를 타이어 양 끝단에 위치시킨다. 비드 와이어는 고무로 코팅된 강철 와이어로 구성되어 있으며, 타이어와 휠 림이 단단히 결합되도록 고정하는 역할을 한다.
  
  \item \textbf{벨트(Belt) 및 트레드(Tread) 적층}: 카카스 위에는 \textbf{벨트 패키지(Belt Package)}가 배치된다. 벨트는 강철 와이어가 포함된 고무층으로, 타이어의 강성을 높이고 주행 시 변형을 억제하는 역할을 한다. 그 위에 \textbf{트레드(Tread)}가 적층된다. 트레드는 노면과 직접 접촉하는 부분으로, 내마모성과 접지력이 뛰어난 고무를 사용한다.
  
  \item \textbf{사이드월(Sidewall) 부착}: 타이어의 측면에 사이드월을 부착한다. 사이드월은 타이어를 외부 충격으로부터 보호하고, 주행 중의 변형에 대응할 수 있도록 유연성이 높은 고무로 제작된다.
  
  \item \textbf{성형 완료}: 모든 층이 드럼에 적층된 후, 드럼이 회전하면서 각 층을 압착하여 일체화한다. 이때 완성된 미가황 타이어를 \textbf{그린 타이어(Green Tire)}라고 하며, 이후 가류공정을 통해 최종 제품으로 완성된다.
\end{enumerate}

\subsection{타이어 가류 공정(Tire Vulcanization Process)}

\subsubsection{가류 공정의 정의 및 목적}

가류(Vulcanization) 공정이란 성형된 타이어를 고온·고압 환경에서 가열하여 고무 분자 사슬 사이에 황(Sulfur) 가교 결합(Cross-link)을 형성시키는 공정이다. 이를 통해 고무는 강도, 내열성, 내마모성, 탄성 복원력을 갖는 구조 재료로 변화한다. 가류 공정은 타이어의 최종 성능과 안전성을 결정하는 핵심 공정이다.

성형 공정을 마친 유연한 타이어를 몰드에 넣고, 열과 압력을 가하는 과정이다. 몰드는 마치 붕어빵을 만들어내는 붕어빵 틀과 같은 기능을 하는 부품으로, 아직 문양 없이 밋밋한 상태의 타이어는 몰드를 만나고 난 뒤 비로소 패턴과 형태가 만들어진다. 따라서 가류 공정은 타이어의 최종 형태를 완성하는 중요한 과정이다.

\section{타이어 배합 설계}

\subsection{타이어 성능의 마법의 삼각형}

\subsubsection{성능 지표의 트레이드-오프}

타이어의 주요 성능 지표는 세 가지 상충하는 요소로 구성되어 있다:

\begin{enumerate}
  \item \textbf{젖은 노면 접지력(제동력)}: 비오는 도로에서의 안전성과 제동 성능
  \item \textbf{내마모성}: 타이어의 수명과 내구성
  \item \textbf{구름 저항(굴림 저항)}: 연비와 주행 효율성
\end{enumerate}

이 세 가지 조건을 충족시키기 위해서는 반비례 관계를 보인다. 한 가지 성능을 개선하려면 다른 성능이 저하되는 경향이 있어, 배합 설계 시 이 3가지 요소를 균형있게 개선하는 것이 가장 큰 관건이다.

\subsubsection{각 성능 지표의 물리적 기제}

\paragraph{젖은 노면 접지력}

제동력은 외부 마찰에 의한 점탄성 손실(Viscoelastic Loss)에 크게 의존한다. 타이어와 노면 간의 접촉에서 변형 및 복원되는 과정이 반복되고 이로 인해 큰 에너지 손실이 발생한다. 이러한 에너지 손실은 노면과 타이어 사이의 강한 상호작용을 의미하며, 손실된 에너지에 비례해서 마찰력이 증가한다. 이를 통해 공회전, 미끄러짐, 차량 흔들림을 방지하여 제동 성능 및 주행 안정성을 향상시킨다. 특히나 젖은 노면에서는 물막에 의해 노면 요철이 평탄해져 타이어와 노면 간의 상호작용이 감소하므로, 점탄성 손실을 통한 접지 확보가 더욱 필요하다.

\paragraph{구름 저항(굴림 저항)}

구름 저항은 타이어 내부에서 발생하는 에너지 손실로, 이는 타이어가 변형 및 복원할 때 일부 복원되지 못한 점탄성 손실에 의해 발생한다. 구름 저항의 원인으로는 하중, 공기압, 온도 등이 있다. 특히나 열의 경우, 주행 중에 타이어 내부에서 분자 사슬이 변형되어 발생한 마찰열로 인해 에너지 손실이 발생한다. 이렇게 발생한 내부적 마찰은 차량의 연비와 주행 성능을 크게 저하시키고, 나아가 연료 낭비로 인한 배기가스 증가로 인해 환경오염을 증대시킨다.

\paragraph{내마모성}

내마모성은 타이어의 자체적인 수명과 관련이 있다. 내마모성이 강할수록 타이어를 더욱 오래 사용할 수 있게 되고, 이를 통해 비용 절감 및 환경 개선에 영향을 준다.

\subsubsection{성능 요소 간의 상충 관계}

각 성능 지표를 개별적으로 증진시키려면 나머지 요소에 영향을 주어 오히려 타이어 성능을 크게 저하시킨다. 예를 들어:

\begin{itemize}
  \item \textbf{젖은 노면 접지력 증가}: 변형 및 복원에 용이해야 하므로 내마모성이 높으면 안 됨 → 굴림 저항 값 증가 → 주행 성능 저하
  
  \item \textbf{내마모성과 굴림 저항의 상충}: 내마모성이 너무 높으면 하중 분포 비균등, 미세 슬립 증가 등으로 인해 저주파 점탄성 손실 증가 → 굴림 저항 값 상승 → 연비 저하
  
  \item \textbf{최적 배합}: 마법의 삼각형의 3요소를 적절히 배분한 배합 조건이 필수
\end{itemize}

\subsubsection{PSR을 이용한 최적 배합 개발}

이 세 가지 요소를 모두 보완한 새로운 배합 조건이 개발되었다. Peijin Weng 등의 연구에서는 기존의 친환경 고성능 타이어(그린 타이어) 배합 조건인 합성수지(Solution Styrene Butadiene Rubber, SSBR)–충전재(실리카)–커플러(TESPT, bis[3-(triethoxysilyl)propyl] tetrasulfide)에 PSR(phosphonium-modified petroleum resin)을 소량 첨가하여, 굴림 저항을 개선하는 동시에 나머지 두 요소에는 영향을 주지 않는 최적 배합을 발견하였다.

\paragraph{실리카 충전계의 문제점}

실리카 충전계에서 가장 큰 문제는 실리카 표면의 실란올기(Si–OH) 간 수소결합으로 인해 발생하는 강한 응집(filler–filler network)이다. 이는 분산 불량을 초래하여 다음을 유발한다:

\begin{itemize}
  \item 열발생 증가
  \item 굴림 저항 악화
  \item 내마모성 저하
\end{itemize}

\paragraph{PSR의 메커니즘}

PSR이 실리카 충전계에 도입되면, PSR의 양전하를 띤 phosphonium 작용기($\mathrm{P^+R_4}$)가 실리카 표면의 음전하를 띤 실란올기(Si–OH)와 정전기적 인력 또는 친핵성 치환 반응을 통해 결합하여 Si–O–P 형태의 새로운 결합 네트워크를 형성한다. 

이 과정의 주요 효과:

\begin{enumerate}
  \item \textbf{실리카-고무 계면 결합 강화}: 기존 TESPT의 황(S$_4$) 기반 커플링 반응과 병행되어 실리카-고무 계면 결합을 크게 강화
  
  \item \textbf{실리카 분산 개선}: 실리카–고무 계면을 균일하게 만들고 충전재 간 응집을 감소시킴
  
  \item \textbf{히스테리시스 손실 감소}: 충전재 간 응집이 줄어들어 히스테리시스 손실($\tan \delta(60 °\mathrm{C})$)을 감소시켜 굴림 저항 개선
  
  \item \textbf{젖은 노면 접지력 유지}: 고무 표면층의 고주파 점탄성 특성은 유지되어 젖은 노면 접지력($\tan \delta(0 °\mathrm{C})$)에는 거의 영향을 주지 않음
  
  \item \textbf{내열성 및 내구성 향상}: 균일해진 충전 네트워크는 고무 내부 열 발생(heat build-up)을 억제
\end{enumerate}

\paragraph{최적 PSR 첨가량의 중요성}

이러한 효과는 PSR의 첨가량이 최적 범위일 때만 안정적으로 나타난다:

\begin{itemize}
  \item \textbf{부족 첨가}: PSR이 적게 첨가되면 실리카 표면과 결합할 수 있는 작용기 밀도가 부족해 응집을 충분히 해체하지 못함
  
  \item \textbf{과다 첨가}: 고무 이외 상(PSR-rich domain)이 형성되면서 plasticization 효과와 미세 응집을 유발해 점탄성 특성 및 기계적 강도를 저하시킴
\end{itemize}

\paragraph{실험 결과와 성과}

최적의 결과를 가져온 실험 결과(PSR 2 phr 첨가):

\begin{itemize}
  \item $\tan \delta(60 °\mathrm{C})$ 약 14\% 감소 → \textbf{굴림 저항 개선}
  \item $\tan \delta(0 °\mathrm{C})$ 약 19\% 증가 → \textbf{젖은 노면 제동력 유지}
  \item 마모 손실 약 28\% 감소 → \textbf{내마모성 유지}
\end{itemize}

이는 단일 첨가제 변경만으로 트레드 물성을 다방면에서 개선한 매우 실용적이고 활용도 높은 사례이다.

\section{타이어 트레드 공압출}

\subsection{3중 공압출 기술 개요}

3중 공압출(tri-composite co-extrusion)은 세 개의 고무 스트림이 단일 다이에서 동시에 성형되는 공정이다. 이를 통해:

\begin{itemize}
  \item 층별 고무 배합의 독립적 최적화 가능
  \item 재료 비용 최소화
  \item 공정 효율성 극대화
\end{itemize}

\subsection{기본 지배방정식 및 구성식}

\subsubsection{연속 방정식과 운동량 방정식}

질량 보존 원리에 따른 연속 방정식은:
\begin{equation}
\frac{D\rho}{Dt} + \rho\nabla \cdot \mathbf{v} = 0
\label{eq:continuity_material}
\end{equation}

비압축 유동($\rho = \text{constant}$)에서는 간단히:
\begin{equation}
\nabla \cdot \mathbf{v}_k = 0; \quad k = I, II, III
\label{eq:continuity_incomp}
\end{equation}

각 층($k = I, II, III$는 크라운, 베이스, 윙)에 대해 독립적으로 질량이 보존되어야 하며, 계면에서도 속도의 법선 성분이 연속이어야 한다.

정상 유동에서의 운동량 방정식:
\begin{equation}
\rho\frac{D\mathbf{v}}{Dt} = \rho\mathbf{g} + \nabla \cdot \boldsymbol{\sigma}
\label{eq:momentum_general}
\end{equation}

이를 다시 쓰면:
\begin{equation}
\rho\left(\frac{\partial \mathbf{v}}{\partial t} + \mathbf{v} \cdot \nabla\mathbf{v}\right) = \nabla \cdot \boldsymbol{\sigma} + \rho\mathbf{g}
\label{eq:momentum_explicit}
\end{equation}

정상 유동($\partial/\partial t = 0$)에서는:
\begin{equation}
\rho_k(\mathbf{v}_k \cdot \nabla)\mathbf{v}_k = -\nabla p_k + \nabla \cdot \boldsymbol{\tau}_k
\label{eq:momentum_steady}
\end{equation}

여기서:
\begin{itemize}
  \item $\rho_k$: 각 층의 밀도
  \item $\mathbf{v}_k$: 속도 벡터
  \item $p_k$: 정수압(hydrostatic pressure)
  \item $\boldsymbol{\tau}_k$: 편차응력(deviatoric stress)
  \item $\rho\mathbf{g}$: 중력의 영향 (수평 압출에서는 무시)
\end{itemize}

응력 텐서는 등방 응력과 편차응력으로 분해된다:
\begin{equation}
\boldsymbol{\sigma}_k = -p_k\mathbf{I} + \boldsymbol{\tau}_k
\label{eq:stress_decomposition}
\end{equation}

\subsubsection{속도 구배의 분해: 변형률과 회전}

속도 구배 텐서 $\nabla\mathbf{v}$는 항상 대칭 성분과 반대칭 성분으로 분해된다:
\begin{equation}
\nabla \mathbf{v} = \mathbf{D} + \mathbf{W}
\label{eq:velocity_gradient_decomp}
\end{equation}

변형률률 텐서(Rate-of-Deformation Tensor, 대칭):
\begin{equation}
\mathbf{D} = \frac{1}{2}\left(\nabla\mathbf{v} + (\nabla\mathbf{v})^T\right)
\label{eq:rate_deformation}
\end{equation}

회전 텐서(Spin Tensor, 반대칭):
\begin{equation}
\mathbf{W} = \frac{1}{2}\left(\nabla\mathbf{v} - (\nabla\mathbf{v})^T\right)
\label{eq:spin_tensor}
\end{equation}

물리적 해석:
\begin{itemize}
  \item $\mathbf{D}$: 유체 요소의 모양 변화 속도 -- 늘어남, 압축, 전단을 나타냄
  \item $\mathbf{W}$: 유체 요소의 강체 회전 속도 -- 좌표계가 회전하는 효과
\end{itemize}

\subsubsection{객관성 원칙 (Objectivity Principle)}

중요한 물리적 원칙: 순수 강체 회전(pure rigid body motion)에서 재료 내부에 응력이 생기면 안 된다.

순수 회전 상황에서:
\begin{itemize}
  \item $\mathbf{D} = 0$ (모양이 변하지 않음)
  \item $\nabla\mathbf{v} = \mathbf{W} \neq 0$ (회전은 있음)
\end{itemize}

따라서 물리적으로 타당한 구성식은 $\mathbf{D}$와 객관적 시간 미분(objective time derivative)에만 의존해야 한다. 회전 텐서 $\mathbf{W}$는 응력의 직접적인 원인이 될 수 없고, 오직 컨벡티드 미분을 통해 간접적으로만 나타난다.

\subsection{PTT 점탄성 모델과 다이 팽윤}

\subsubsection{편차응력의 분해 및 구성식}

점탄성 유체에서 전체 편차응력 $\boldsymbol{\tau}$는 탄성 성분 $\boldsymbol{\tau}_1$과 점성 성분 $\boldsymbol{\tau}_2$로 분해된다:
\begin{equation}
\boldsymbol{\tau} = \boldsymbol{\tau}_1 + \boldsymbol{\tau}_2
\label{eq:stress_decomp_viscoelastic}
\end{equation}

\begin{itemize}
  \item $\boldsymbol{\tau}_1$: \textbf{탄성 성분(Elastic Component)} - 과거 변형의 이력을 기억하여 정상응력 생성
  \item $\boldsymbol{\tau}_2$: \textbf{점성 성분(Viscous Component)} - 뉴턴 유체처럼 현재 변형률에만 의존
\end{itemize}

점성 성분:
\begin{equation}
\boldsymbol{\tau}_2 = 2\eta_2 \mathbf{D}
\label{eq:tau_2}
\end{equation}

\subsubsection{탄성 성분과 PTT 모델}

파라미터 $\eta_1, \lambda, \varepsilon, \xi$를 가진 PTT(Phan-Thien-Tanner) 구성식:
\begin{equation}
\exp\left[\frac{\varepsilon\lambda}{\eta_1}\mathrm{tr}(\boldsymbol{\tau}_1)\right]\boldsymbol{\tau}_1 + \lambda\left[(1-\tfrac{\xi}{2})\overset{\nabla}{\boldsymbol{\tau}_1} + \tfrac{\xi}{2}\overset{\triangle}{\boldsymbol{\tau}_1}\right] = 2\eta_1\mathbf{D}
\label{eq:ptt}
\end{equation}

여기서:
\begin{itemize}
  \item $\eta_1$: 탄성 점도(elastic viscosity) - 고무 고유의 점도 특성
  \item $\lambda$: 이완시간(relaxation time) - 고분자 사슬의 완화 시간, 초 단위
  \item $\varepsilon$: 신장 거동 파라미터 - 신장응력에 대한 민감도
  \item $\xi$: 전단 거동 파라미터 - 전단 모드의 특성 조절
  \item $\mathrm{tr}(\boldsymbol{\tau}_1)$: 응력 텐서의 자취(trace) - $\tau_{11} + \tau_{22} + \tau_{33}$
\end{itemize}

\subsubsection{비선형 응력 구성식의 의미}

\paragraph{지수함수 항의 역할}

\begin{equation}
\exp\left[\frac{\varepsilon\lambda}{\eta_1}\mathrm{tr}(\boldsymbol{\tau}_1)\right]
\label{eq:exp_term}
\end{equation}

이 항은 다음을 담당한다:
\begin{enumerate}
  \item \textbf{신장응력 기억}: 고분자 사슬이 신장될 때 응력이 지수적으로 증가
  \item \textbf{비선형성}: 강한 전단에서는 응력이 비선형으로 변함
  \item \textbf{응력 포화}: 과도한 신장에서 응력 증가가 제한됨 (고분자 사슬의 한계)
\end{enumerate}

\paragraph{컨벡티드 미분과 회전 효과}

상-컨벡티드 미분(Upper-Convected Derivative):
\begin{equation}
\overset{\triangle}{\boldsymbol{\tau}_1} = \frac{D\boldsymbol{\tau}_1}{Dt} - \boldsymbol{\tau}_1 \cdot (\nabla\mathbf{v})^T - (\nabla\mathbf{v}) \cdot \boldsymbol{\tau}_1
\label{eq:upper_convected_explicit}
\end{equation}

하-컨벡티드 미분(Lower-Convected Derivative):
\begin{equation}
\overset{\nabla}{\boldsymbol{\tau}_1} = \frac{D\boldsymbol{\tau}_1}{Dt} + \boldsymbol{\tau}_1 \cdot (\nabla\mathbf{v})^T + (\nabla\mathbf{v})^T \cdot \boldsymbol{\tau}_1
\label{eq:lower_convected_explicit}
\end{equation}

이 두 미분의 가중 평균:
\begin{equation}
(1-\tfrac{\xi}{2})\overset{\nabla}{\boldsymbol{\tau}_1} + \tfrac{\xi}{2}\overset{\triangle}{\boldsymbol{\tau}_1}
\end{equation}

은 응력의 좌표계 회전에 대한 객관성(objectivity)을 보장하면서도, 매개변수 $\xi$를 통해 특정 고무의 거동에 맞도록 조절한다.

\subsubsection{다이 팽윤의 물리적 메커니즘}

PTT 구성식 (\ref{eq:ptt})의 좌변:
\begin{equation}
\text{(지수함수)} \times \text{(응력)} + \text{(이완시간)} \times \text{(응력의 시간변화)}
\end{equation}

우변:
\begin{equation}
2\eta_1 \mathbf{D} = \text{(점도)} \times \text{(현재 변형률)}
\end{equation}

물리적 의미: 응력의 시간 진화는 현재의 변형률(우변)에 의해 구동되지만, 과거에 축적된 응력(좌변의 지수함수)과 응력의 이완(좌변의 컨벡티드 미분)에 의해 조절된다.

\subsubsection{다이 팽윤율과 정상응력}

\subsubsection{다이 팽윤의 물리적 메커니즘}

다이 팽윤(die swell)은 압출 출구에서 고분자 사슬의 탄성 회복과 속도 재분배에 의해 발생하는 현상이다. 발생 과정:

\begin{enumerate}
  \item \textbf{다이 내부}: 신장응력 축적, 지수함수 항 강화
  \item \textbf{출구 직후}: 구속 해제, 탄성 에너지 방출, 횡 방향 팽창
  \item \textbf{원인}: 정상응력 차이 $N_1 = \tau_{11} - \tau_{22} > 0$
\end{enumerate}

\paragraph{다이 팽윤율}

\subsubsection{다이 팽윤의 물리적 메커니즘}

다이 팽윤율(die swell ratio, swelling index):
\begin{equation}
B = \frac{S - S_0}{S_0}\times 100\% = \frac{\Delta S}{S_0} \times 100\%
\label{eq:swell}
\end{equation}

여기서:
\begin{itemize}
  \item $S_0$: 다이 출구의 단면적
  \item $S$: 압출 후 어느 거리에서 측정한 단면적 (보통 다이 출구로부터 수 cm)
  \item $B$: 백분율로 표현한 팽윤율
\end{itemize}

예를 들어 $B = 20\%$는 출구 단면이 다이 폭의 20% 더 팽창했다는 의미이다.

\subsubsection{PTT 파라미터와 다이 팽윤의 관계}

같은 다이와 공정 조건에서도 재료에 따라 팽윤이 달라진다. PTT 파라미터의 영향:

\begin{equation}
B = f(\lambda, \eta_1, \varepsilon, \dot\gamma)
\label{eq:swell_ptt}
\end{equation}

\begin{itemize}
  \item \textbf{이완시간 $\lambda$ 증가}: 응력 완화가 느려짐 → 팽윤 증가
  \item \textbf{신장 파라미터 $\varepsilon$ 증가}: 신장응력에 대한 민감도 증가 → 팽윤 증가
  \item \textbf{점도 $\eta_1$ 증가}: 응력 크기 증가 → 팽윤 증가 (전체 흐름 저항 증가)
\end{itemize}

특히 3중 공압출에서는 세 층(크라운, 베이스, 윙)의 $\lambda, \varepsilon$ 값이 서로 다르므로, 각 층의 팽윤율이 상이하다:
\begin{equation}
B_{\text{Crown}} \neq B_{\text{Base}} \neq B_{\text{Wing}}
\label{eq:swell_three_layers}
\end{equation}

이러한 팽윤 차이로 인해 계면 형상이 왜곡되고, 최종 제품의 치수 편차와 형상 불일치가 발생한다.

\subsection{유동 균일성과 설계 파라미터 효과}

\subsubsection{유동 균일성 지수와 벽면 전단률}

3중 공압출에서 세 층의 계면 형상은 각 층의 유동 특성에 따라 결정된다. 유동 불균일성(flow non-uniformity):
\begin{equation}
\delta = \frac{v_{\max} - v_{\min}}{v_{\text{avg}}} \times 100\%
\label{eq:flow_uniformity}
\end{equation}

$\delta$가 작을수록:
\begin{itemize}
  \item 출구에서 속도 분포가 균일
  \item 계면 형상이 더 정규적
  \item 제품 두께 편차 감소
  \item 표면 품질 향상
\end{itemize}

\paragraph{벽면 전단률}

다이 내부 벽면에서의 전단률:
\begin{equation}
\dot\gamma_w = \left.\frac{dv}{dy}\right|_{\text{wall}}
\label{eq:wall_shear_rate}
\end{equation}

높은 벽면 전단률의 문제점:
\begin{itemize}
  \item 고분자 주쇄의 기계적 절단(shear-induced chain scission)
  \item 재료 분해(thermal degradation) 가속
  \item 표면 결함 발생(표면 주름, 거칠기 증가)
  \item 압출 품질 저하
\end{itemize}

따라서 공정 설계에서는 $\dot\gamma_w$를 허용 범위 이하로 제어해야 한다.

\subsection{설계 파라미터와 공정 최적화}

\subsubsection{다이 너비와 수렴각}

\paragraph{다이 너비 (Die Width)}

다이 너비 $W$가 증가할 때:
\begin{equation}
\frac{\partial \delta}{\partial W} < 0, \quad \frac{\partial \dot\gamma_w}{\partial W} < 0
\end{equation}

즉, 너비 증가 → 속도 분산 감소 → 유동 균일성 향상. 동시에 벽면 전단률이 감소하여 표면 결함 위험도 낮아진다. 그러나 다이 너비가 과도하게 증가하면 제조 비용과 다이 구조적 안정성 문제가 발생한다.

\subsubsection{다이 수렴각 (Die Convergence Angle)}

다이 수렴각을 $\alpha$라 할 때:
\begin{equation}
\frac{\partial \dot\gamma_w}{\partial \alpha} > 0, \quad \frac{\partial \delta}{\partial \alpha} \gtrless 0
\end{equation}

수렴각 증가 → 벽면 전단률 급증 → 표면 품질 저하. 동시에 압출 방향에 따라 유동 분배가 급격하게 변하는 고전단 영역이 입구에서 출구로 이동하면서 형상 편차도 증가한다.

\subsubsection{유량과 인발 속도}

유입 유량 $Q$와 관련된 특성 전단률 $\dot\gamma_c = Q/A$ (여기서 $A$는 다이 단면적)라 하면:

다이 내부 평균 전단률:
\begin{equation}
\dot\gamma = \frac{Q}{A} \cdot f(\text{geometry})
\end{equation}

유량 증가 → 전단률 증가 → PTT 모델의 비선형성 강화. 식 (\ref{eq:ptt})의 지수함수 항이 강해지면서:
\begin{itemize}
  \item 다이 팽윤 증가 (식 (\ref{eq:swell}))
  \item 벽면 전단률 증가 (표면 결함 위험)
  \item 유동 균일성 저하
\end{itemize}

\paragraph{인발 속도}

인발 속도를 $v_d$라 할 때, 다이 입구와 출구 사이의 응력 평형 관계:
\begin{equation}
\frac{\partial B}{\partial v_d} < 0
\end{equation}

즉, 인발 속도 증가 → 다이 팽윤 감소. 이는 출구 이후의 응력 완화 시간이 단축되기 때문이다.

\subsubsection{계면 전단과 윙팁 영역}

세 개 층의 PTT 파라미터가 상이할 때, 경계층 근처에서 강한 전단이 집중된다:
\begin{equation}
\dot\gamma_{\text{interface}} > \dot\gamma_{\text{bulk}}
\end{equation}

특히 윙팁(wing tip) 영역은 기하학적으로 유로가 좁아 다음이 성립한다:
\begin{equation}
\dot\gamma_w^{\text{wing}} \gg \dot\gamma_w^{\text{bulk}}
\end{equation}

이로 인해 윙팁 영역에 표면 결함(흐름 불안정성, 표면 주름 등)이 선택적으로 발생한다.

\subsection{다중 목적 최적화와 계면 동역학}

\subsubsection{다중 목적 최적화와 파레토 해}

공압출 공정의 설계는 상충(conflict)하는 여러 목표를 동시에 만족해야 한다:

\begin{equation}
\min \mathbf{f}(\mathbf{x}) = \begin{bmatrix}
f_1: & \text{형상 편차} & \delta \\
f_2: & \text{벽면 전단률} & \dot\gamma_w \\
f_3: & \text{층간 팽윤 차이} & |B_i - B_j|
\end{bmatrix}
\label{eq:multi_objective}
\end{equation}

설계 변수(design variables):
\begin{equation}
\mathbf{x} = \begin{bmatrix} W & \alpha & Q & v_d \end{bmatrix}^T
\label{eq:design_variables}
\end{equation}

제약조건(constraints):
\begin{equation}
\begin{cases}
\dot\gamma_w \leq \dot\gamma_{\text{crit}} & \text{(표면 결함 방지)} \\
\delta \leq \delta_{\text{max}} & \text{(유동 균일성 요구)} \\
B \leq B_{\text{max}} & \text{(제품 두께 편차 제어)} \\
Q \geq Q_{\text{min}} & \text{(생산성 보장)} \\
v_d \leq v_{\text{max}} & \text{(냉각 능력 고려)}
\end{cases}
\label{eq:constraints}
\end{equation}

\paragraph{파라미터 감도 분석}

각 설계 변수가 성능 메트릭에 미치는 영향을 정량화하기 위해 감도 분석(sensitivity analysis)을 수행한다:

\begin{equation}
S_{ij} = \frac{\partial f_i}{\partial x_j} \bigg|_{\mathbf{x}_0}
\label{eq:sensitivity}
\end{equation}

예를 들어:

\begin{itemize}
  \item \textbf{다이 너비 W에 대한 유동 균일성 감도}:
  \begin{equation}
  S_{\delta,W} = \frac{\partial \delta}{\partial W} < 0 \quad \text{(너비 증가 → 균일성 향상)}
  \end{equation}
  
  \item \textbf{수렴각 $\alpha$에 대한 벽면 전단률 감도}:
  \begin{equation}
  S_{\dot\gamma_w,\alpha} = \frac{\partial \dot\gamma_w}{\partial \alpha} > 0 \quad \text{(수렴각 증가 → 전단률 증가)}
  \end{equation}
  
  \item \textbf{유량 Q에 대한 다이 팽윤 감도}:
  \begin{equation}
  S_{B,Q} = \frac{\partial B}{\partial Q} > 0 \quad \text{(유량 증가 → 팽윤 증가)}
  \end{equation}
\end{itemize}

\paragraph{파레토 최적해와 실무 적용}

다중 목적 최적화에서는 여러 파레토 최적해가 존재한다. 이는 한 목표를 개선하려면 다른 목표를 악화시켜야 하는 상황을 의미한다:

\begin{equation}
\text{파레토 최적해} \Leftrightarrow \neg \exists \mathbf{x}' \in \mathcal{F}: f_i(\mathbf{x}') \leq f_i(\mathbf{x}) \, \forall i
\label{eq:pareto_optimality}
\end{equation}

여기서 $\mathcal{F}$는 가능(feasible) 영역이다. 실제 공정 선택은 가중 스칼라화(weighted scalarization)를 통해 단일 목적함수로 변환한다:

\begin{equation}
\min \sum_{i=1}^3 w_i f_i(\mathbf{x}), \quad \sum w_i = 1, \, w_i \geq 0
\label{eq:weighted_sum}
\end{equation}

\subsubsection{계면 형상 결정}

세 층의 경계면 위치는 다음 원칙으로 결정된다:

\begin{enumerate}
  \item \textbf{질량 보존}: 각 층의 체적 유량 연속성
  \begin{equation}
  Q_k = \int_{\Omega_k} v_n(\mathbf{r}) \, dA, \quad k = I, II, III
  \label{eq:volume_conservation}
  \end{equation}
  
  \item \textbf{응력 평형}: 계면에서 응력 연속성
  \begin{equation}
  \boldsymbol{\tau}_k^{\text{left}} \cdot \mathbf{n} = \boldsymbol{\tau}_{k+1}^{\text{right}} \cdot \mathbf{n}
  \label{eq:stress_continuity}
  \end{equation}
  
  \item \textbf{속도 연속성}: 법선 방향 속도가 같음
  \begin{equation}
  v_{n,k} = v_{n,k+1} \quad \text{at interface}
  \label{eq:velocity_continuity}
  \end{equation}
\end{enumerate}

\subsection{층간 상호 작용}

각 층의 PTT 파라미터가 다르면, 유동 특성에 큰 차이가 난다:

\begin{equation*}
\begin{aligned}
\text{Crown}: &\quad \lambda_C, \varepsilon_C \; \Rightarrow\; B_C = f(\lambda_C, \varepsilon_C, \dot\gamma_C) \\
\text{Base}: &\quad \lambda_B, \varepsilon_B \; \Rightarrow\; B_B = f(\lambda_B, \varepsilon_B, \dot\gamma_B) \\
\text{Wing}: &\quad \lambda_W, \varepsilon_W \; \Rightarrow\; B_W = f(\lambda_W, \varepsilon_W, \dot\gamma_W)
\end{aligned}
\end{equation*}

일반적으로 $B_C > B_B > B_W$이므로, 계면이 곡선 형태로 왜곡된다:

\begin{equation}
\text{계면 곡률} \propto |B_i - B_j| \quad \Rightarrow \text{계면 형상 왜곡}
\label{eq:interface_curvature}
\end{equation}

\subsection{수치해석 방법론 개요}

3중 공압출 PTT 점탄성 유동 해석은 식 (\ref{eq:ptt})이 포함된 강한 비선형 편미분방정식 시스템이다. 본 보고서에서는 상용 CFD 솔버인 ANSYS Polyflow 기준으로 다음 절차로 해석한다고 기술한다.

\begin{enumerate}
  \item \textbf{격자}: 다이 입출구/벽면 근처 자동 세분화, 계면 추적을 위한 적응 메시
  \item \textbf{물성}: 각 층의 PTT 파라미터 $(\eta_1,\lambda,\varepsilon,\xi)$ 입력, 밀도/점도/온도 조건 설정
  \item \textbf{경계 조건}: 입구 유량 또는 압력, 출구 압력 또는 인발 속도, 벽면 무미끄럼
  \item \textbf{해석}: 안정화된 FEM + 뉴턴-랍슨 반복으로 수렴, 필요 시 안정화 옵션 활용
  \item \textbf{결과}: 다이 팽윤, 층간 계면 형상, 속도/전단률/정상응력 분포 후처리
\end{enumerate}

계산 시간은 격자 규모(약 5만~30만 요소)와 수렴 조건에 따라 수 시간~수십 시간이 소요될 수 있다.

  이를 컴팩트하게 쓰면 다음과 같은 비선형 지배방정식 시스템으로 표현된다:

  \begin{equation}
  \mathcal{G}(\mathbf{v}, p, \boldsymbol{\tau}_1) = 
  \begin{cases}
  \nabla \cdot \mathbf{v} = 0 \\[4pt]
  \rho (\mathbf{v} \cdot \nabla) \mathbf{v} = -\nabla p + \nabla \cdot (2\eta_0 \mathbf{D} + \boldsymbol{\tau}_1) \\[4pt]
  \exp\Big[\frac{\varepsilon \lambda}{\eta_1} \mathrm{tr}(\boldsymbol{\tau}_1)\Big] \boldsymbol{\tau}_1 + \lambda \Big[(1-\tfrac{\xi}{2})\overset{\nabla}{\boldsymbol{\tau}_1} + \tfrac{\xi}{2}\overset{\triangle}{\boldsymbol{\tau}_1}\Big] = 2\eta_1 \mathbf{D}
  \end{cases}
  \label{eq:governing_full}
  \end{equation}

\paragraph{최적화 문제의 실제 적용}

다이 너비, 수렴각, 유량, 인발 속도를 설계 변수로 하는 다중 목적 최적화:

\begin{equation}
\begin{array}{ll}
\min & J = w_1 \delta_{\text{outlet}} + w_2 \max(\dot\gamma_w) + w_3 \max(|B_i - B_j|) + w_4 Q_{\text{cost}} \\
\text{subject to} & Q_{\min} \leq Q \leq Q_{\max} \\
& \alpha_{\min} \leq \alpha \leq \alpha_{\max} \\
& W_{\min} \leq W \leq W_{\max} \\
& v_d \leq v_{d,\max}
\end{array}
\label{eq:optimization_practical}
\end{equation}

설계 가중치는 공정의 우선순위에 따라 다음과 같이 설정한다. 형상 정밀도 우선($w_1 = 0.3$), 표면 품질 우선($w_2 = 0.4$), 계면 형상 제어($w_3 = 0.2$), 경제성($w_4 = 0.1$).

\subsection{공정 설계 지침 및 최종 결론}

\paragraph{주요 연구 성과}

타이어 트레드 3중 공압출 공정에 대한 본 연구는 다음과 같은 성과를 도출하였다. 첫째, PTT 점탄성 모델을 통해 타이어 트레드 3중 공압출 유동을 정확히 모사할 수 있음을 보였다. 둘째, 다이 팽윤이 응력 텐서와 정상응력에 의존함을 명시적으로 도출했다. 셋째, 층별 팽윤 차이가 계면 형상 왜곡의 직접적 원인임을 보였다. 넷째, 다이 너비와 수렴각이 유동 균일성과 벽면 전단률에 미치는 정량적 영향을 정식화했다. 다섯째, 다중 목적 최적화 제약 조건 하에서 파레토 최적해를 도출하는 방법론을 제시했다.

\subsubsection{다이 너비 선택}

다이 너비가 커질수록 출구 형상 편차가 완만히 감소한다는 경험식을 사용한다:

\begin{equation}
\delta(W) \approx \delta_0 \left(\frac{W_0}{W}\right)^{n_\delta}, \quad n_\delta \in [0.6, 0.9]
\label{eq:die_width_effects}
\end{equation}

\begin{equation}
W^* = \arg\min_W \left(\delta(W) + \lambda_{\text{cost}} C(W)\right)
\label{eq:optimal_width}
\end{equation}

권장사항:
\begin{itemize}
  \item 형상 정밀도 우선: $W$를 크게 설정 ($\geq 100$ mm)
  \item 표면 품질 중시: $W$를 중간 범위 유지 (50-80 mm)
  \item 비용 우선: $W$를 작게 설정 ($\leq 50$ mm) - 대신 다이 내부 피드 블록 설계 최적화
\end{itemize}

\subsubsection{수렴각 선택}

수렴각이 커지면 벽면 전단률이 거의 선형적으로 증가한다:

\begin{equation}
\dot\gamma_w(\alpha) \approx \dot\gamma_{w,0} \left(1 + k_\alpha \tan \alpha\right), \quad k_\alpha \approx 0.8{-}1.2
\label{eq:convergence_angle_effects}
\end{equation}

벽면 전단률 제약:
\begin{equation}
\alpha_{\max} = \arctan\left(\frac{\dot\gamma_{\text{crit}} \cdot h}{v_{\text{exit}}}\right)
\label{eq:max_convergence}
\end{equation}

권장사항:
\begin{itemize}
  \item 표면 품질 우선: $\alpha = 5°$-$10°$ (완만한 수렴, 낮은 전단)
  \item 균형 설계: $\alpha = 10°$-$15°$ (적당한 전단, 합리적 다이 길이)
  \item 비용 최적화: $\alpha = 15°$-$20°$ (높은 전단, 짧은 다이) - 재료 제어 필수
\end{itemize}

\subsubsection{유입 유량 설정}

유량을 높이면 다이 팽윤이 완만히 증가한다는 근사식을 사용한다:

\begin{equation}
B(Q) \approx B_0 \left[1 + k_Q \left(\frac{Q}{Q_0} - 1\right)\right], \quad k_Q \sim 0.3{-}0.6
\label{eq:flow_rate_effects}
\end{equation}

이를 이용해 목표 형상에 맞춘 보정식을 정의한다 (식 \ref{eq:flow_rate_effects} 기반):

\begin{equation}
Q_{\text{opt}} = Q_{\text{nominal}} \times \left(1 - \frac{\delta_{\text{target}} - \delta_{\text{actual}}}{\delta_{\text{max}}}\right)
\label{eq:flow_tuning}
\end{equation}

권장사항:
\begin{itemize}
  \item 생산 목표량 > 형상 정밀도: $Q = Q_{\text{max}}$ (고속 생산, 정밀도 보정)
  \item 형상 정밀도 > 생산량: $Q = 0.7 \times Q_{\text{max}}$ (저속 운전, 안정성 우선)
  \item 중간 타협: $Q = 0.85 \times Q_{\text{max}}$ (생산성과 품질 병행)
\end{itemize}

\subsubsection{인발 속도 최적화}

인발 속도가 증가하면 외부 팽윤이 감소한다는 실험적 추세를 반영한다:

\begin{equation}
B(v_d) \approx B_{\text{free}} - k_v \left(\frac{v_d}{v_{\text{ref}}} - 1\right), \quad k_v \sim 0.4{-}0.7
\label{eq:draw_speed_swell}
\end{equation}

식 (\ref{eq:draw_speed_swell})의 역의 관계 활용:

\begin{equation}
v_d = \frac{Q}{S_0} \cdot \kappa, \quad \kappa = 1 + \frac{B_{\text{target}} - B_{\text{actual}}}{B_{\text{max}}}
\label{eq:draw_speed_opt}
\end{equation}

권장사항:
\begin{itemize}
  \item 다이 팽윤 과다 ($B > 15\%$): $v_d$ 증가 (인발 속도 2-3\% 상향)
  \item 다이 팽윤 부족 ($B < 5\%$): $v_d$ 감소 (인발 속도 2-3\% 하향)
  \item 적정 범위 ($5\% < B < 15\%$): $v_d$ 유지
\end{itemize}

\paragraph{고급 제어 전략 및 실시간 모니터링}

각 층의 유량을 독립적으로 제어하기 위해 피드 블록 설계를 다음과 같이 한다:

\begin{equation}
h_k^{\text{feed}} = \frac{Q_k \cdot L}{v_{\text{nominal}} \cdot W}
\label{eq:feed_block_height}
\end{equation}

여기서 $k = \text{Crown}, \text{Base}, \text{Wing}$이다. 실시간 모니터링을 위해 다이 입구·출구의 압력 센서, 다이 내부의 온도 센서, 압출 출구의 카메라, 제품 인발력 센서를 설치하여 흐름 저항, 점도 변화, 형상 편차, 응력 상태를 추정한다. 피드백 제어는 다음 식으로 표현된다:

\begin{equation}
\Delta Q = K_P (B_{\text{target}} - B_{\text{actual}}) + K_I \int (B_{\text{target}} - B_{\text{actual}}) \, dt
\label{eq:feedback_control}
\end{equation}

\paragraph{향후 연구 과제}

향후 연구 과제로는 다음을 제시한다. 현재 2D 또는 축대칭 해석이 주류이므로, 완전 3D 다이 형상에서의 이방성 효과를 고려한 수치해석이 필요하다. 또한 점도의 온도 의존성, 전단 가열, 냉각 효과를 포함한 열 이송 해석을 추가해야 한다. PTT 모델을 넘어 multi-mode 점탄성 모델이나 비선형 비탄성 모델을 적용하여 더 높은 정확도를 추구해야 한다. 전단 화이트닝(shear whitening) 현상과 표면 결함의 근본 원인을 규명하기 위한 표면 불안정성 분석도 필요하며, 수치해석 결과와 실제 압출 실험의 체계적 비교를 통해 모델을 검증해야 한다.



\section{타이어 가류 공정}

\subsection{가류 메커니즘}

\subsubsection{가류 반응 속도 모델 (Arrhenius 식)}

가류 반응은 온도 의존 화학 반응이며, 반응 속도는 Arrhenius 식으로 표현된다:

\begin{equation}
k = A \exp\left(-\frac{E_a}{RT}\right)
\label{eq:arrhenius}
\end{equation}

여기서:
\begin{itemize}
  \item $k$: 반응 속도 상수
  \item $A$: 전지수 인자(Pre-exponential factor)
  \item $E_a$: 활성화 에너지(Activation energy)
  \item $R$: 기체 상수 (8.314 J/(mol·K))
  \item $T$: 절대 온도(K)
\end{itemize}

온도가 증가하면 반응 속도는 지수적으로 증가하며, 온도가 과도하면 과가류(Over-cure)가 발생해 오히려 강도가 감소한다.

\subsubsection{가교 밀도와 기계적 물성 관계}

가교 밀도(ν)는 고무의 탄성계수(E)와 비례 관계를 가진다:

\begin{equation}
E = 3\nu RT
\label{eq:modulus_crosslink}
\end{equation}

가교 밀도가 증가할수록 강성과 내마모성이 증가한다. 단, 과도한 가교는 취성 증가와 피로 수명 감소를 초래한다.

\subsubsection{가류 진행률 적산 모델}

가류 진행률(Cure conversion, α)은 반응 속도의 시간 적분으로 표현되며, 동일한 가류 시간이라도 온도에 따라 가류 정도가 크게 달라진다:

\begin{equation}
\alpha(t) = \int_0^t k(T) \, dt
\label{eq:cure_conversion}
\end{equation}

$\alpha = 0$일 때는 미가류(Uncured), $\alpha = 1$일 때는 완전 가류(Fully cured) 상태를 의미한다.

\subsection{가류 조건 최적화}

\subsubsection{가류 공정의 열전달 지배 방정식}

가류는 금형에서 고무 내부로 열이 전달되는 열전도 지배 공정이다:

\begin{equation}
\rho c \frac{\partial T}{\partial t} = k \nabla^2 T
\label{eq:heat_equation}
\end{equation}

여기서:
\begin{itemize}
  \item $\rho$: 고무의 밀도
  \item $c$: 비열용량(Specific heat)
  \item $k$: 열전도율(Thermal conductivity)
  \item $\nabla^2 T$: 온도의 라플라시안
\end{itemize}

타이어는 두께가 크기 때문에 표면과 내부의 가류 정도가 서로 달라질 수 있다. 이를 열 이동 수(Fourier number)로 표현할 수 있다:

\begin{equation}
Fo = \frac{k t}{\rho c L^2}
\label{eq:fourier}
\end{equation}

여기서 $L$은 특성 길이(타이어 두께의 절반)이다.

\subsubsection{산업 현장 가류 조건}

일반적인 타이어 가류 조건:
\begin{itemize}
  \item \textbf{온도}: 150 ~ 170°C
  \item \textbf{압력}: 15 ~ 22 bar
  \item \textbf{시간}: 10 ~ 20 분
  \item \textbf{냉각 시간}: 3 ~ 5 분
\end{itemize}

\subsubsection{가류 불량의 수식적 정의}

\begin{equation}
\text{미가류(Under-cure)}: \quad \alpha < \alpha_{\text{opt}} \quad \Rightarrow \quad \text{강도 부족, 조기 마모}
\label{eq:undercure}
\end{equation}

\begin{equation}
\text{과가류(Over-cure)}: \quad \alpha > \alpha_{\text{opt}} \quad \Rightarrow \quad \text{취성 증가, 균열 발생}
\label{eq:overcure}
\end{equation}

최적 가류 조건에서는 $\alpha \approx \alpha_{\text{opt}}$ (보통 0.90 ~ 0.95)일 때 최고의 기계적 성능을 나타낸다.

\paragraph{온도-시간 최적화}

에너지 최소화와 가류도 최적화를 동시에 고려:

\begin{equation}
\min J = w_1 \int_0^{t_c} T(t) \, dt + w_2 (1 - \alpha(t_c))^2
\label{eq:optimization}
\end{equation}

여기서 $w_1, w_2$는 가중치, $t_c$는 가류 시간이다.

\paragraph{내부 미가류 방지}

금형 예열 및 블래더 압력 균일화가 필수적이다. Fourier 수가 충분히 커야 타이어 전체가 균일하게 가열된다:

\begin{equation}
Fo > 0.3 \quad \Rightarrow \quad t_c > \frac{0.3 \rho c L^2}{k}
\label{eq:min_time}
\end{equation}

\subsection{가류 특성 평가}

\subsubsection{가류 공정의 공정능력(Cp, Cpk) 평가}

가류온도, 시간, 압력의 안정성을 공정능력 지수로 평가한다:

\begin{equation}
C_p = \frac{USL - LSL}{6\sigma}
\label{eq:cp}
\end{equation}

\begin{equation}
C_{pk} = \min\left(\frac{USL - \mu}{3\sigma}, \frac{\mu - LSL}{3\sigma}\right)
\label{eq:cpk}
\end{equation}

여기서:
\begin{itemize}
  \item $USL$: 상한 규격(Upper Specification Limit)
  \item $LSL$: 하한 규격(Lower Specification Limit)
  \item $\mu$: 평균값
  \item $\sigma$: 표준편차
\end{itemize}

$C_{pk} \geq 1.33$ 이상이면 공정이 안정하다고 판단한다.

\subsubsection{가류 후 검사 및 후처리}

가류가 완료된 타이어는 시각적 검사와 물리적 검사를 거친다. 외관의 결함, 트레드 패턴의 균일성, 균열 여부 등을 확인하며, 필요 시 X-레이 검사로 내부 결함을 검출한다. 이후 불필요한 고무 잔여물을 제거하고, 로고 및 규격 표시를 인쇄한 뒤 출하된다.

\paragraph{타이어 스퓨(수염) 발생 원인}

바로 이 가류 공정 중 고무를 몰드에 찍는 과정에서 자연스럽게 스퓨가 발생한다. 고무재료를 몰드에 넣어 타이어를 만들 때, 몰드 내부의 공기가 밖으로 빠져 나오도록 미세한 구멍을 뚫게 된다. 바로 이 구멍으로 고무가 흘러나와서 굳은 것이 스퓨(타이어 수염)가 된다. 따라서 스퓨는 타이어 성능과는 전혀 관계가 없으며, 시간이 지나면 자연스럽게 제거된다.

몰드 내부는 압력이 높을 뿐만 아니라, 부위별로 온도가 다를 수도 있기 때문에 구멍을 뚫지 않으면 고무의 쏠림 현상이 발생할 수 있다. 또한 가류 공정 중 가하는 압력, 시간, 온도에 따라 전혀 다른 특성의 타이어가 만들어질 수 있기 때문에 이 때 고도의 기술과 노하우가 필요하다. 현재는 이와 같은 스퓨가 제거되지 않고 그대로 출고되었지만, 최근에는 깨끗하게 제거하는 트리밍 공정 기술을 통해 스퓨가 없는 타이어 또한 생산되고 있다.

\subsection{온도·압력 프로파일 최적화}

\subsubsection{기존 가황 공정의 문제점}

가황 반응은 Arrhenius 식 (식 \ref{eq:arrhenius})에 따라 온도 변화에 매우 민감하다. 금형 내 온도 편차, 일정 압력 유지 방식, 그리고 시간·온도·압력의 상호작용 제어 부재는 가류 불균일의 주요 원인이 된다. 또한 가류 진행률 α(t)=∫k(T)dt 관계식(식 \ref{eq:cure_conversion})에서 보이듯, 동일 시간이라도 온도 조건에 따라 가류도가 크게 달라진다.

\subsubsection{온도 프로파일 최적화 (Variable Temperature Curing)}

온도 프로파일 최적화는 가황 단계를 초기–중간–후반의 3단계로 구분해 제어하는 방식이다:

\begin{itemize}
  \item \textbf{초기 고온 단계}: 반응 속도 k(T)을 빠르게 증가시켜 가교반응을 빠르게 시작하며 내부 가류 지연을 완화한다. 이를 통해 전체 가황 시간을 약 10~15\% 단축할 수 있다.
  
  \item \textbf{중간 안정 단계}: 금형 온도 분포가 균일해지는 구간으로, 설계된 가교 밀도 확보에 유리하다.
  
  \item \textbf{후반 저온 단계}: 과가류를 방지하고 물성 저하를 막으며, 트레드 패턴 손상을 최소화한다.
\end{itemize}

\subsubsection{압력 프로파일 최적화}

압력은 고무 흐름성, 패턴 형성, 기포 제거에 중요한 영향을 미친다:

\begin{itemize}
  \item \textbf{초기 저압 단계}: 고무의 유동성을 확보해 금형 내부 충전 성능을 향상시킨다.
  
  \item \textbf{증압 단계}: 가교가 진행되면서 고무 점도가 증가하므로 압력을 점진적으로 높여 패턴 형성과 내부 결합을 강화한다.
  
  \item \textbf{최종 고압 유지}: 패턴 선명도를 확보하고 보강재와 고무의 결합력을 높여 제품 내구성을 향상시킨다.
\end{itemize}

\subsubsection{온도–압력 연동 최적화 (PT Coupled Control)}

온도와 압력을 독립적으로 제어하는 방식에서 벗어나, 두 변수를 상호 연동하여 제어하는 전략이다. 가류 진행률은 다음과 같이 표현된다:

\begin{equation}
\alpha(t) = \int_0^t f(T, P) \, dt
\label{eq:cure_tp_coupled}
\end{equation}

여기서 $f(T,P)$는 온도와 압력의 상호작용에 따른 반응률이다. 예를 들어 온도가 높은 경우 필요한 압력을 낮춰도 동일한 가류 효과를 얻을 수 있으며, 온도가 낮은 경우 압력 증압 곡선을 조정하여 균일한 가류 품질을 유지할 수 있다.

이러한 온도-압력 연동 제어를 통해 다음의 개선 효과를 얻을 수 있다:

\begin{itemize}
  \item 내부·외부 가류 균일도 증가
  \item 과가류/미가류 불량 감소
  \item 전체 생산성 향상 (가황 시간 10~15\% 단축)
  \item 패턴 선명도 및 외관 품질 향상
  \item 스퓨(Spew) 발생량 감소
  \item 공정능력(Cpk) 향상 및 품질 안정성 강화
\end{itemize}


\section{결론}

\subsection{주요 성과}

\begin{itemize}
  \item \textbf{공정-성능 연계 체계화}: 정련부터 가류·검사까지의 공정 체인을 균일성(RFV/LFV 등) 및 표면·내구 지표와 직결되는 변수-결과 지도로 정리하여, 개별 공정 최적화가 최종 품질에 미치는 경로를 명확화했다.

  \item \textbf{공압출 물리 메커니즘 정식화}: PTT 모델을 통해 다이 팽윤(정상응력 기인)·계면 왜곡(층별 $\lambda, \varepsilon$ 차)·벽면 전단률을 설명하고, $W, \alpha, Q, v_d$의 변화가 $\delta, \dot{\gamma}_w, |B_i-B_j|$에 주는 정량적 경향을 감도식으로 제시했다.

  \item \textbf{설계·운전 지침 도출}: 형상 정밀·표면 품질·경제성을 아우르는 다중목적 최적화 관점에서 다이 너비·수렴각·유량·인발속도의 선택 가이드를 제안하고, \textbf{인라인 모니터링(압력·온도·형상)–피드백 제어}($\Delta Q, v_d$)의 적용 틀을 정리했다.

  \item \textbf{가류 균일화 전략 제시}: Arrhenius 반응–열전달 모델로 미/과가류 경계를 정의하고, 가열 3단(초기 고온–중간 안정–후반 저온) 및 압력 단계 제어, PT 연동 제어의 효과(내·외부 가류 균일, $C_{pk}$ 향상, 사이클 단축)를 논증했다.
\end{itemize}

\subsection{한계와 개선 방안}

\begin{itemize}
  \item \textbf{모델 축약성}: 단일/소수 모드 PTT 및 등온 가정 등으로 인해 절대치 예측보다는 경향 파악에 적합하다. 열-유동 완전 연성, 다중모드·비탄성 모델 적용이 필요하다.

  \item \textbf{스코프 제약}: 분석 초점이 트레드 공압출·가류에 치우쳐 있으며, 캘린더링·성형 편차의 정량 반영이 제한적이다. 전 공정의 통합 시뮬레이션과 라인 데이터 연계가 요구된다.

  \item \textbf{현장 검증 데이터}: 온·압·형상·인발력의 동시 로깅과 $C_p/C_{pk}$ 기반 통계검증으로 파라미터 식(경향식)의 교정이 필요하다.
\end{itemize}

\subsection{최종 결론}

본 보고서는 공압출–가류 중심의 공정 물리를 기반으로 설계 변수 $\to$ 유동/반응 $\to$ 형상·계면 $\to$ 품질지표로 이어지는 인과 사슬을 정리하고, 다중목적 최적화+피드백 제어로 의사결정을 구조화하였다. 이후 현장 데이터에 의한 모델 보정, 3D 다이·열연성 해석, 배합–캘린더–성형을 포함한 엔드투엔드 통합 모델, 경제·안전성 동시 평가가 병행될 때, 제시한 프레임은 고품질·고효율 생산을 지향하는 실무 기준으로 확장될 수 있다.

\section*{보고서 진행 웹페이지}

본 과제의 진행 상황은 다음 웹페이지에서 실시간으로 확인할 수 있습니다:

\begin{center}
  \Large\textbf{타이어 공정설계 과제 진행사항}\\
  \url{https://roids7.github.io/tire-report/tire_report.html}
\end{center}

\noindent 위 링크에서는 배합·공압출·가류·보고서 작성 등 각 단계별 완료 현황과 상세 내용을 확인할 수 있습니다.

\section*{참고문헌}

\begin{enumerate}
  \item Wang, G.-L., Zhou, H.-J., Zhou, H.-C., Liang, C., ``Viscoelastic Numerical Simulation Study on the Co-Extrusion Process of Tri-Composite Tire Tread,'' \textit{Materials}, 16(9), 3301 (2023).
  \item Jeong, K. M., Kim, K. W., Beom, H. G., and Park, J. U., ``Finite Element Analysis of Nonuniformity of Tires with Imperfections,'' \textit{Tire Science and Technology}, Vol. 35, No. 3, pp. 226--238, 2007.
  \item Weng, P., Tang, Z., Guo, B., ``Solving `magic triangle' of tread rubber composites with phosphonium-modified petroleum resin,'' \textit{Polymer}, 190, 122244 (2020).
  \item Mark, J. E., Erman, B., \& Roland, C. M. (2013). \textit{The Science and Technology of Rubber}. Academic Press. (고무 가교 메커니즘, 가류 반응 모델, 열·물성 변화)
  \item Stephens, H., \& Bhowmick, A. K. (2001). \textit{Handbook of Elastomers}. CRC Press. (가류 공정의 물성 변화, 배합·가교 밀도와 기계적 특성 관계)
  \item Coran, A. Y. (1995). Vulcanization. In \textit{Science and Technology of Rubber}. (Arrhenius 기반 가류 반응속도 모델, 최적 가황 조건 분석)
  \item Montgomery, D. C. (2009). \textit{Introduction to Statistical Quality Control}. Wiley. (공정능력(Cp, Cpk) 분석 이론적 근거)
  \item \textit{Rubber Extrusion Handbook}, Hanser Publishers.
  \item ``Effect of Vulcanization Conditions on Mechanical Properties of Tire Rubber,'' Elsevier.
  \item ``Calendering Process of Tire Cord Rubber Composites,'' ScienceDirect.
  \item ``Arrhenius Kinetics in Vulcanization: Temperature-Dependent Cross-Linking Dynamics,'' \textit{Rubber Chemistry and Technology}, Vol. 82, No. 5, 2009.
  \item ``Heat Transfer and Cure Kinetics in Tire Vulcanization,'' \textit{Journal of Applied Polymer Science}, 128(2), 2013.
  \item Michelin / Bridgestone Tire Manufacturing Technical Report.
\end{enumerate}

\end{document}
